\documentclass[pdflatex,sn-mathphys-num]{sn-jnl}
\usepackage[T1,T2A]{fontenc}
\usepackage[utf8]{inputenc}
\usepackage[english,russian]{babel}
\usepackage{amsmath}
\makeatletter
\AtBeginDocument{\let\latexlabel\label}
\makeatother

\title[Пасталогия]{Пасталогия: Аксиоматическая теория мироздания}

\author*[1]{\fnm{Владимир} \sur{Ситников}}\email{montenegrofsm@google.com}

\affil*[1]{\orgdiv{Независимый исследователь}, \city{Бар}, \country{Черногория}}

\abstract{Настоящая статья представляет собой изложение пасталогии – новой аксиоматической теории мироздания, разработанной на основе единственного допущения: материя делима, но не до бесконечности. Из этого простого аксиоматического принципа дедуктивно выводятся все остальные положения теории, включая природу фундаментальных частиц – пастонов, концепции пространства и времени, а также объяснение эмерджентных явлений. В рамках пасталогии гравитация интерпретируется как эффект давления вездесущих фоновых пастонов, свет – как упорядоченные сгустки пастонов, а корпускулярно-волновой дуализм объясняется взаимодействием частиц с интерферирующим фоновым волновым фронтом. Космологические загадки, такие как тёмная материя и тёмная энергия, разрешаются без введения дополнительных сущностей: тёмная материя объясняется наличием крупных пастонов и их агрегатов, а тёмная энергия отвергается за ненадобностью, поскольку ускоренное расширение Вселенной не предполагается. Теория предлагает ряд проверяемых предсказаний, отличающихся от предсказаний стандартной физики, включая нетождественность гравитационной и инертной масс для сверхплотных объектов, специфику сигналов ``гравитационных волн'' от сливающихся сверхплотных систем и сохранение интерференции в опытах со щелями даже при гарантированном знании пути частицы. Пасталогия позиционируется не как раздел физики, а как самостоятельная аксиоматическая наука, стремящаяся к построению полной и непротиворечивой картины мира из его фундаментальных логических основ.}

\keywords{аксиоматическая теория, фундаментальная физика, гравитация, корпускулярно-волновой дуализм, космология}

\begin{document}

\maketitle

\section{Введение}\label{sec:introduction}

\subsection{Мотивация и методология исследования}\label{subsec:motivation-methodology}

Современная физика, несмотря на свои неоспоримые успехи, сталкивается с рядом фундаментальных проблем и неясностей, которые ставят под сомнение полноту и непротиворечивость текущих парадигм. Существование неэлиминируемых ``полей'' как первичных сущностей, сложности в синтезе общей теории относительности \cite{einstein1916} и квантовой механики \cite{bohr1928}, а также необходимость введения гипотетических ``необъяснимых сущностей'', таких как тёмная материя \cite{rubin1980} и тёмная энергия \cite{riess1998}, указывают на потенциальные пробелы в нашем понимании мироздания. Эти вызовы побуждают к поиску более фундаментальных, логически стройных и аксиоматических объяснений устройства Вселенной.

Настоящая работа представляет пасталогию – новую аксиоматическую науку, которая предлагает радикально иной подход к познанию реальности. В отличие от эмпирической физики, которая движется ``снизу вверх'' --- от наблюдений и экспериментов к формулировке законов и моделей, --- пасталогия следует дедуктивному пути ``сверху вниз'', начиная с единственного допущения и логически выводя из него всю картину мироздания. Этот подход аналогичен аксиоматическим системам в математике, таким как геометрия Евклида \cite{euclid-elements-heath}, где из нескольких постулатов выстраивается целая система знаний.

Мы рассматриваем Вселенную как своего рода ``чёрный ящик'' \cite{wiener1948} --- огромный и сложный объект с неизвестными механизмами реализации. Физика стремится понять его поведение через анализ наблюдаемых реакций, что соответствует классическому эмпирическому подходу в науке \cite{popper1959}. Пасталогия же пытается ``заглянуть внутрь'' этого ящика, предположив самые базовые принципы его устройства и затем дедуктивно вывести, как эти принципы порождают наблюдаемые явления. В перспективе мы предполагаем, что должна произойти ``встреча'' пасталогии и физики, когда эмпирические данные, полученные физикой, сойдутся с логическими выводами пасталогии, формируя единую, полную и непротиворечивую картину мира.

\subsection{Цель работы}\label{subsec:purpose}

Основной целью данной статьи является представление пасталогии как новой аксиоматической теории, полностью выведенной из единственного допущения. Мы стремимся продемонстрировать логическую строгость и последовательность дедуктивных выводов, которые позволяют объяснить широкий круг физических явлений и космологических загадок, зачастую предлагая альтернативные интерпретации по сравнению со стандартными моделями.

\subsection{Введение терминологии}\label{subsec:terminology}

Для адекватного изложения аксиоматической основы пасталогии вводятся новые, специфические термины, напрямую заимствованные из пастафарианского мировоззрения. Фундаментальные, неделимые частицы, составляющие всю материю, названы ``пастонами'' (в отличие от ``атомов'' Демокрита \cite{kirk1983-democritus} или ``монад'' Лейбница \cite{leibniz1989-monadology}, поскольку пасталогия имеет собственное обоснование природы этих частиц). Также выделена отдельная категория пастонов --- ``тефтоны''. Тефтон --- это пастон достаточно большого размера, чтобы испытывать воздействие фоновых пастонов меньшего размера, в результате которого тефтоны могут группироваться. Особое место занимают ``тефтели'' --- \textbf{пупермассивные} чёрные дыры, которые в контексте пасталогии обладают уникальной способностью порождать новые локальные ``Вселенные'', подобные нашей наблюдаемой ``Вселенной'', путём взаимного столкновения и разрушения. Остальные термины, например, такие как ``гравитационная тень'' или ``гравитационная вязкость'', будут объяснены в тексте статьи.

Выбор данной терминологии и названия самой науки -- ``пасталогия'' -- неслучаен и имеет принципиальное методологическое значение. В отличие от теологии, которая, претендуя на научный статус, не имеет под собой научного метода и опирается на веру и ``духовный опыт'', пасталогия предлагает строго аксиоматический, логически выводимый и научно проверяемый подход к пониманию мироздания. Таким образом, через иронию и аллюзию на мировоззрение, изложенное в Евангелии Летающего Макаронного Монстра \cite{henderson2006}, мы утверждаем, что сама Вселенная, со всеми её законами и явлениями, и есть проявление Летающего Макаронного Монстра, представленное через систему детерминированных взаимодействий пастонов. Эта ``материалистически-религиозная'' концепция не умаляет научности пасталогии, но подчёркивает её претензию на фундаментальное объяснение реальности, выводящееся из простейших начал, пользуясь принципом бритвы Оккама \cite{ockham-razor-sep}.

\subsection{Структура статьи}\label{subsec:structure}

Настоящая статья организована следующим образом: раздел~\ref{sec:fundamentals} описывает фундаментальные принципы пасталогии, включая её единственное аксиоматическое допущение и вытекающие из него свойства пастонов. Раздел~\ref{sec:space-time} посвящен интерпретации пространства и времени в рамках теории. Раздел~\ref{sec:emergent-cosmology} излагает эмерджентные явления и космологические следствия, такие как природа гравитации, света, разрешение ключевых космологических проблем и корпускулярно-волнового дуализма. Раздел~\ref{sec:predictions} представляет собой перечень проверяемых следствий и предсказаний пасталогии, некоторые из которых напрямую противоречат положениям стандартной физики. Заключительный раздел~\ref{sec:discussion-conclusion} содержит обсуждение сильных сторон теории, её места в научной парадигме и перспективы будущих исследований.

\section{Фундаментальные принципы пасталогии}\label{sec:fundamentals}

\subsection{Единственное допущение (аксиома)}\label{subsec:axiom}

В основе пасталогии лежит единственное фундаментальное допущение, служащее аксиомой, из которой дедуктивно выстраивается вся последующая картина мироздания. Этот аксиоматический принцип сформулирован следующим образом: материя делима, но не до бесконечности. Это допущение опирается на фундаментальное убеждение пасталогии в познаваемости мира и причинности как основе логики и реальности, что отличает теорию от подходов, отрицающих смысл научного исследования.

Данное допущение признаёт очевидное, наблюдаемое свойство материи --- её делимость. Мы можем последовательно делить объекты на всё более мелкие компоненты: вещество делится на молекулы, молекулы на атомы, атомы на элементарные частицы. Однако в отличие от концепций, предполагающих бесконечное деление, пасталогия постулирует наличие конечного предела этой делимости. Это означает, что существует некий неделимый и бесструктурный элемент, при достижении которого дальнейшее деление невозможно. Именно этот элементарный, неделимый объект и является фундаментом всей материи во Вселенной.

\subsection{Природа фундаментальных частиц --- пастонов}\label{subsec:pastons}

На основе аксиомы о конечной делимости материи мы приходим к определению фундаментальной частицы: пастона.

Пастон: истинно элементарный, бесструктурный объект. Он не состоит из каких-либо меньших компонентов и не может быть разделён. Это означает, что у пастона отсутствует внутренняя динамика, обусловленная движением его составных частей, а все его свойства являются внутренними и неотъемлемыми. В отличие от других концепций, постулирующих фундаментальные частицы без внутренней структуры (например, лептоны или кварки в Стандартной Модели \cite{pdg2024}), пасталогия не приписывает пастонам сложных свойств, таких как электрический заряд, спин или цвет, как их фундаментальных атрибутов. Такие сложные свойства могут быть обусловлены только сложностью строения, и в пасталогии они интерпретируются как эмерджентные характеристики, возникающие из специфических конфигураций и взаимодействий множества пастонов в более сложных агрегациях.

Исходя из того, что пастоны могут взаимодействовать друг с другом, логически следует, что они обладают размером и формой. Взаимодействие происходит через столкновения, как будет показано далее, что невозможно для точечных объектов. Допускается, что пастоны могут обладать различной формой и размером, что в дальнейшем может объяснить разнообразие их агрегаций.

Ключевым свойством пастона является его инертная масса. В рамках пасталогии инертная масса определяется как внутреннее, неотъемлемое свойство самого пастона, напрямую пропорциональное его объёму (размеру). Для любого составного объекта (будь то элементарная частица, атом или планета) его инертная масса является просто суммой инертных масс всех составляющих его пастонов.

Поскольку пастоны являются бесструктурными и неделимыми, из этого логически следует их неизменность и вечность. Пастон не может быть создан или уничтожен, он существует вне времени и не подвержен распаду. Это в свою очередь приводит к выводу о вечности и бесконечности Вселенной, которая состоит из этих вечных и неизменных пастонов.

Между столкновениями пастоны движутся прямолинейно и равномерно. Изменение их движения возможно только в результате взаимодействия с другими пастонами. Понятие импульса, как произведение инертной массы пастона на его скорость, является фундаментальным для описания их движения и взаимодействий.

\subsection{Единственное фундаментальное взаимодействие}\label{subsec:interaction}

Поскольку пастоны являются бесструктурными, неделимыми и не могут обладать внутренними силами или полями, единственным физически возможным способом их взаимодействия является абсолютно упругий удар. Это означает, что при столкновении двух пастонов происходит идеальное сохранение импульса без каких-либо потерь энергии, переходящей во внутренние состояния (поскольку внутренних состояний у бесструктурного пастона быть не может). Возможное вращение пастона, возникшее при столкновении, также не ведёт к потере энергии, поскольку вся энергия вращения полностью возвращается в кинетическую энергию при последующих столкновениях.

Все прочие наблюдаемые силы и взаимодействия в природе --- электрические, магнитные, ядерные --- не являются фундаментальными. В пасталогии они интерпретируются как эмерджентные, статистические эффекты, возникающие из огромного количества последовательных абсолютно упругих столкновений пастонов, организованных в более сложные агрегации (такие как электроны, протоны и другие частицы, обычно именуемые фундаментальными).

\subsubsection{Передача импульса при ударе пастонов}\label{subsubsec:impulse-transfer}

Классическая механика рассматривает абсолютно упругий удар составных тел, обладающих внутренней структурой. В этом случае процесс включает две фазы: (1)~сжатие тел при сближении, когда кинетическая энергия частично превращается во внутреннюю энергию деформации; (2)~восстановление формы, при котором внутренняя энергия возвращается в кинетическую, обеспечивая характерный \emph{отскок}, формулируемый как закон равенства угла падения и угла отражения. Таким образом, классическая модель удара опирается на существование внутренней ``пружины'' в материальном теле \cite{feynman-lectures-v1}.

Для пастонов, как неделимых и бесструктурных элементов материи, такая модель неприменима. Пастон не имеет внутренних степеней свободы, которые могли бы аккумулировать и возвращать энергию удара. Следовательно, удар пастонов представляет собой \emph{моментальную передачу импульса} без стадии деформации и восстановления. В результате:

\begin{enumerate}
  \item Время контакта пастонов при столкновении равно нулю.
  \item Отсутствует эффект отражения: импульс всегда передаётся по направлению движения ударяющего пастона.
  \item При равных массах движущийся пастон полностью останавливается в момент удара, а второй пастон продолжает движение с его скоростью по параллельной траектории из собственных координат.
  \item При различии масс распределение импульса подчиняется закону сохранения, но без возможности ``пружинного возврата'': более массивный пастон получает меньшую скорость, обратно пропорциональную его массе, а менее массивный не может превысить скорость ударяющего.
\end{enumerate}

Таким образом, в пасталогии \textbf{абсолютно упругий удар} понимается не как упругий отскок, а как прямая и мгновенная передача импульса между бесструктурными элементами. Это отличие от классической механики принципиально и должно учитываться во всех дальнейших построениях. Следует отметить, что, поскольку пастоны обладают размером и формой, нецентральные столкновения неизбежно будут приводить к возникновению вращения. Эта вращательная степень свободы усложняет динамику взаимодействий, так как кинетическая энергия может перераспределяться между поступательным и вращательным движением. Однако это не нарушает принципа абсолютно упругого удара, так как общая энергия системы сохраняется. Более того, именно эта фундаментальная возможность вращения может лежать в основе такого наблюдаемого свойства частиц, как спин, а также определять специфику их взаимодействия с асимметричными потоками фоновых пастонов (что проявляется, например, в особенностях их траекторий вблизи крупных агрегатов материи). Детальный анализ динамики вращающихся пастонов выходит за рамки данной обзорной статьи, но является ключевым направлением для будущих исследований.

\subsection{Детерминизм и невычислимость}\label{subsec:determinism}

Пасталогия является строго детерминированной теорией на микроуровне. Движение каждого отдельного пастона и результат каждого столкновения полностью предопределены начальными условиями и законами упругого удара. Однако из-за бесконечного количества пастонов во Вселенной и невообразимого числа их последовательных взаимодействий, поведение системы на макроуровне становится невычислимым. Эта невычислимость порождает кажущуюся случайность, которую мы наблюдаем в квантовых явлениях, но это не является истинной индетерминированностью, а лишь следствием сложности и невозможности отслеживания всех индивидуальных взаимодействий.

\subsection{Энергия}\label{subsec:energy}

В пасталогии энергия рассматривается как математическая абстракция, а не как самостоятельная физическая сущность или субстанция. Она представляет собой способ описания характеристик движения и процессов, происходящих с пастонами. Кинетическая энергия, в частности, является количественной мерой движения пастонов и их способности выполнять работу через столкновения.

\section{Пространство и время в пасталогии}\label{sec:space-time}

\subsection{Пространство}\label{subsec:space}

В традиционной физике пространство часто рассматривается как некая самостоятельная сущность, обладающая собственными свойствами и способная искривляться или расширяться \cite{mtw1973}. Однако в пасталогии предлагается радикально иное понимание. Пространство определяется как ``Ничто'' -- это не физический объект, не поле, не ``ткань'' и не субстанция. Оно представляет собой лишь место существования материи, то есть пастонов.

Пространство, по своей сути, не обладает никакими собственными физическими свойствами. Оно не имеет массы, энергии, плотности, кривизны или какой-либо внутренней структуры. Следствием такого определения является невозможность для пространства само по себе ``искривляться'' или ``расширяться''. Наблюдаемые эффекты, которые в других теориях интерпретируются как деформация пространства-времени (например, гравитационные явления или космологическое красное смещение), в пасталогии объясняются через изменения в плотности, распределении или динамике самих пастонов.

Между пастонами -- абсолютная пустота. Это означает, что пространство не заполнено ничем, кроме пастонов. Любое взаимодействие происходит только через прямой абсолютно упругий удар между пастонами. Таким образом, пасталогия предлагает Вселенную, где есть только дискретная материя (пастоны) и пустое пространство между ними, а все наблюдаемые явления являются следствием их механического движения и столкновений.

\subsection{Время}\label{subsec:time}

Подобно пространству, время в пасталогии является математической абстракцией, а не самостоятельной физической сущностью или измерением. Время не ``течёт'', не ``существует'' и не обладает собственными свойствами или направлением.

Вместо этого, время определяется как мера последовательности изменений и характеристик процессов, происходящих с пастонами. Мы воспринимаем время как ``текущее'' потому, что наблюдаем череду движений и взаимодействий пастонов. ``Прошлое'', ``настоящее'' и ``будущее'' -- это исключительно ментальные конструкты, возникающие из нашего восприятия последовательности событий. В реальности существует лишь непрерывно меняющаяся конфигурация Вселенной, состоящая из пастонов, находящихся в постоянном движении и взаимодействии.

Отсутствие физического времени означает, что нет ``мест'' во времени, куда можно было бы переместиться, и нет ``потока'', который мог бы замедляться или ускоряться сам по себе или в силу каких-либо причин. Все эффекты, которые в других теориях связываются с замедлением времени (например, релятивистское замедление), должны быть объяснены в пасталогии как изменения в частоте или скорости физических процессов, происходящих с пастонами, а не как изменение свойств самого времени.

\section{Эмерджентные явления и космология в пасталогии}\label{sec:emergent-cosmology}

\subsection{Гравитация как эффект давления}\label{subsec:gravity-pressure}

В пасталогии гравитация не рассматривается как фундаментальная сила взаимодействия или искривление пространственно-временного континуума. Вместо этого она интерпретируется как эмерджентный эффект давления, возникающий из множественных абсолютно упругих столкновений.

На первый взгляд, данный механизм может показаться схожим с историческими теориями ``push-гравитации'', наиболее известной из которых является теория Жоржа-Луи Ле Сажа \cite{lesage-edwards2014}. Однако различие между пасталогией и моделью Ле Сажа является фундаментальным и методологическим. Теория Ле Сажа была гипотезой, предложенной \textit{ad hoc} специально для объяснения гравитации, и требовала постулирования особых ``ультрамировых корпускул'' со специфическими свойствами (например, абсолютно неупругими столкновениями, что приводило к неразрешимой проблеме колоссального разогрева тел). В пасталогии же гравитация как эффект давления не является исходным постулатом, а представляет собой \textbf{неизбежное дедуктивное следствие} из единственной аксиомы о конечной делимости материи. Свойства пастонов и их абсолютно упругие столкновения --- это не допущения для объяснения гравитации, а логические выводы из фундаментального принципа. Таким образом, гравитация в пасталогии --- это не цель теории, а эмерджентное явление, органично возникающее из её основ.

Вселенная пронизана бесчисленным множеством фоновых пастонов, которые находятся в постоянном движении, перемещаясь во всех возможных направлениях. Эти фоновые пастоны сталкиваются с любыми объектами, состоящими из других пастонов.

Эффект, ошибочно традиционно именуемый ``притяжением'', в пасталогии объясняется как дисбаланс давления, создаваемый потоком фоновых пастонов. Когда два или более объекта, состоящих из пастонов, находятся вблизи друг друга, они начинают взаимно ``экранировать'' друг друга от части потока фоновых пастонов, идущих извне. Каждый объект отбрасывает ``гравитационную тень'' --- область, куда фоновые пастоны проникают в меньшем количестве или с меньшей интенсивностью из-за столкновений с пастонами экранирующего объекта.

Таким образом, на каждый объект со всех сторон действует воздействие фоновых пастонов. Однако со стороны другого объекта, находящегося поблизости, это воздействие оказывается чуть меньше из-за взаимного экранирования. Результирующий дисбаланс воздействий приводит к сближению объектов, создавая наблюдаемый эффект гравитации.

Понятие ``гравитационной тени'' тела является ключевым. Это совокупная площадь проекций всех пастонов, составляющих тело, на перпендикулярную потоку фоновых пастонов плоскость. Величина этой тени, пропорциональная инертной массе объекта, определяет, насколько эффективно тело экранирует фоновые пастоны. Однако на близких расстояниях и при анализе воздействия на объекты, находящиеся внутри других тел (например, на поверхности или в глубине планеты), необходимо учитывать пространственное распределение гравитационных теней от составляющих тело пастонов. Это распределение влияет на результирующее гравитационное воздействие.

Эта концепция гравитации как давления, порождённого абсолютно упругими столкновениями, согласуется с принципами механики и позволяет избежать введения необоснованных ``сил'' или ``полей'', что соответствует минималистичному подходу пасталогии.

\subsection{Нетождественность гравитационной и инертной масс}\label{subsec:mass-nonequivalence}

В стандартной физике инертная и гравитационная массы считаются тождественными --- это принцип эквивалентности \cite{einstein1920-relativity}. В пасталогии, однако, для определённых условий инертная и гравитационная массы не являются строго тождественными, что является прямым следствием механизма гравитации как давления фоновых пастонов (см. раздел~\ref{subsec:gravity-pressure}).

Инертная масса объекта, как уже было определено (см. раздел~\ref{subsec:pastons}), представляет собой сумму инертных масс всех составляющих его пастонов. Это мера ``количества материи'' и её сопротивления изменению движения.

Термин ``гравитационная масса'' в пасталогии некорректен и не используется, поскольку в гравитационном эффекте масса как таковая вообще не играет никакой роли. Эффект гравитации обусловлен исключительно геометрическими свойствами пастонов --- их способностью создавать ``гравитационную тень'' или эффективное сечение для столкновений с фоновыми пастонами. Корреляция между этим геометрическим свойством и инертной массой наблюдается лишь в рыхлых телах и является закономерностью, а не причинно-следственной связью. Таким образом, в пасталогии, когда речь идёт о массе, всегда имеется в виду исключительно инертная масса, ибо другой не существует.

Для целей объяснения эффекта, традиционно связываемого с гравитационной массой, используется понятие способности экранирования. Эта способность определяется тем, насколько объект перехватывает поток фоновых пастонов, создавая ``гравитационное давление'' на другие объекты.

Для обычных, не слишком плотных объектов способность экранирования, создаваемая каждым пастоном, суммируется, и совокупный эффект действительно примерно пропорционален общему количеству пастонов, то есть инертной массе. Однако ситуация кардинально меняется при рассмотрении сверхплотных объектов, таких как нейтронные звёзды или чёрные дыры \cite{shapiro-teukolsky1983}.

Внутри таких объектов пастоны расположены чрезвычайно плотно. Это приводит к явлению взаимного экранирования пастонов. Фоновые пастоны, проникающие в такой сверхплотный объект, могут сталкиваться с одними пастонами, прежде чем достигнут других, более глубоко расположенных пастонов. В результате, некоторые внутренние пастоны оказываются экранированы от внешнего потока фоновых пастонов не только другими объектами снаружи, но и своими же ``соседями'' внутри того же самого сверхплотного агрегата. Гравитационные тени пастонов, входящих в состав сверхплотных объектов, накладываются друг на друга, приводя к уменьшению совокупной гравитационной тени.

Следствием этого взаимного экранирования является то, что совокупная эффективность экранирования сверхплотного тела оказывается меньше, чем сумма эффективностей экранирования всех составляющих его пастонов, если бы они находились в разрежённом состоянии. Таким образом, несмотря на то что инертная масса объекта (общее количество пастонов) остаётся неизменной, его способность создавать гравитационное давление (``гравитационный эффект'') уменьшается относительно ожидаемой пропорциональности.

Логический вывод из этого механизма заключается в том, что эффективность экранирования сверхплотных объектов не пропорциональна их инертной массе и будет меньше ожидаемой. Этот эффект становится тем более выраженным, чем выше плотность и больше размеры объекта. Это предсказание является одним из ключевых отличий пасталогии от принципа эквивалентности общей теории относительности и предлагает проверяемый способ фальсификации теории (см. раздел ~\ref{subsec:mass-inequivalence}).

\subsection{Математическая модель гравитации}\label{subsec:gravity-model}

В пасталогии гравитация интерпретируется как эмерджентный эффект давления изотропного потока фоновых пастонов, создающего асимметрию давления из-за экранирования телами. Ключевая величина --- эффективная гравитационная тень тела \( A \) (м²), то есть площадь его проекции на фронт потока, определяющая, насколько тело перехватывает фоновый поток.

Для двух тел с тенями \( A_1 \) и \( A_2 \), инертной массой спутника \( M_2 \) и расстоянием между ними \( r \) (где \( r \) много больше размеров тел), ускорение спутника описывается формулой:
\[
a_2(r) = \frac{4\pi p_0 A_1 A_2}{M_2 r^2}
\]
где \( p_0 \) --- фоновое давление потока пастонов (Н/м²). Эта формула отражает, что гравитационный эффект определяется геометрическими свойствами теней, а не массой как таковой.

Для рыхлых тел, где \( A \propto M \), модель воспроизводит ньютоновский закон тяготения \cite{newton1687-principia, halliday2013-fundamentals} с эффективной константой:
\[
G_{\text{eff}} = 4\pi p_0 k_\ell^2
\]
где \( k_\ell \) --- коэффициент пропорциональности между \( A \) и массой \( M \). Настраивая \( p_0 \) и \( k_\ell \) так, чтобы \( G_{\text{eff}} = G_N \), модель согласовывалась с наблюдениями для обычных объектов.

Для сверхплотных объектов (например, чёрных дыр) зависимость \( A(M) \) становится нелинейной, например,
\[
A \propto M^\alpha, \quad \alpha < 1
\]
в первом приближении \( \alpha \approx \frac{2}{3} \). Это приводит к тому, что гравитационная масса таких объектов всегда меньше их инертной массы, что прямо противоречит строгому принципу эквивалентности в общей теории относительности и даёт проверяемое предсказание (см. раздел~\ref{subsec:mass-inequivalence}).

Пути дальнейшего развития модели включают:
\begin{itemize}
    \item уточнение зависимости \( A(M) \) через численные симуляции \cite{newman2013-computational} (например, Монте-Карло для столкновений пастонов или 3D N-body моделирование агрегатов),
    \item определение параметров \( p_0 \) и \( \alpha \) из наблюдений (сравнение орбитальных ускорений рыхлых и компактных трассеров на одинаковых расстояниях от центра масс),
    \item разработку байесовских методов для подгонки параметров по данным астрофизических наблюдений (например, орбиты звёзд и пульсаров у Sgr A* \cite{ghez2008-sgrA}, гравитационное линзирование \cite{schneider2006-lensing} или будущие данные LISA \cite{lisa-consortium2017}).
\end{itemize}
Эти шаги позволят уточнить количественные предсказания и повысить эмпирическую проверяемость теории.

\subsection{Гравитационные волны}\label{subsec:gravitational-waves}

В классической физике ``гравитационные волны'' интерпретируются как возмущения метрики пространства-времени, распространяющиеся со скоростью света \cite{ligo2016-detection}. В пасталогии, где пространство и время не являются физическими сущностями, такая интерпретация неприемлема. Вместо этого наблюдаемые сигналы, детектируемые, например, интерферометрами LIGO/Virgo, объясняются как периодические изменения совокупной гравитационной тени, порождаемые динамикой сверхплотных объектов.

Как было показано в разделе ~\ref{subsec:gravity-pressure}, гравитация является результатом давления фоновых пастонов, а эффективность этого давления зависит от геометрического экранирования пастонов, составляющих объект. В случае систем из двух или более сверхплотных тел, таких как нейтронные звёзды или чёрные дыры, находящихся на тесных орбитах и активно сближающихся друг с другом (например, в процессе слияния), их взаимное расположение непрерывно изменяется.

Когда два таких сверхплотных объекта вращаются друг вокруг друга, их совокупная экранирующая способность, проецируемая в направлении наблюдателя, расположенного в плоскости орбиты или близко к ней, периодически изменяется. В определённые моменты времени, когда их гравитационные тени максимально перекрываются с точки зрения линии визирования наблюдателя, экранирующая способность системы минимальна. И наоборот, когда перекрытие минимально или отсутствует, совокупная тень максимальна. Интенсивность сигнала при этом зависит от угла наблюдения, достигая максимума в плоскости орбиты и уменьшаясь при отклонении от неё.

Эти периодические изменения в экранирующей способности порождают изменения в гравитационном давлении фоновых пастонов, распространяющиеся от системы вращающихся тел к удалённому наблюдателю. Именно эти изменения в локальном давлении фоновых пастонов регистрируются гравитационными интерферометрами и интерпретируются как ``гравитационные волны''. Они вызывают микроскопические, но периодические деформации детекторов (например, интерферометров), что соответствует наблюдаемым сигналам.

Важно подчеркнуть, что это статистический эффект, обусловленный изменением геометрии распределения и перекрытия пастонов в движущихся сверхплотных системах. Это не является ``волнами в ткани пространства-времени'', а скорее возмущениями в локальной плотности или интенсивности потока фоновых пастонов, вызванными динамикой экранирующей системы (сверхплотных объектов). Это объяснение согласуется с принципом отсутствия силового взаимодействия на уровне пастонов (см. раздел ~\ref{subsec:gravity-pressure}) и эмерджентной природой всех сложных взаимодействий.

\subsection{Природа света и ``старение фотона''}\label{subsec:light-aging}

В пасталогии свет не является дуальной сущностью (волной и частицей одновременно) \cite{debroglie1924-thesis} и не трактуется как колебание электромагнитного поля в пустоте \cite{maxwell1873-treatise}. Его природа объясняется через механику пастонов. 

\paragraph{Фотон как сгусток пастонов.} 
Фотон представляет собой упорядоченный сгусток пастонов, движущийся согласованно в одном направлении. При излучении атомом (точнее, электроном в его составе) часть фоновых пастонов организуется в устойчивую конфигурацию и получает энергию, которая далее переносится как фотон. Механизм организации пока не известен, но принципиально важен результат: фотон — это не ``особая'' частица, а упорядоченный коллектив обычных пастонов.

\paragraph{Механизм ``рокировки''.}
При движении через космос фотон сталкивается с хаотически движущимися фоновыми пастонами. В результате абсолютно упругого удара происходит обмен импульсами: 
\begin{enumerate}
  \item Пастон из состава фотона передаёт импульс фоновому пастону и покидает сгусток.
  \item Новый пастон принимает импульс и встраивается в когерентное движение фотона, но начинает свой путь со своей исходной позиции в пространстве.
  \item Каждый такой акт слегка изменяет положение пастонов в составе сгустка: одни оказываются ближе к центру, другие — на таком же расстоянии от центра, а некоторые — дальше. В результате в среднем часть пастонов постепенно смещается наружу, и линейные размеры сгустка размываются, несмотря на сохранение параллельности траекторий.
\end{enumerate}

\paragraph{Красное смещение как следствие.}
Поскольку в пасталогии длина волны света соотносится с линейным размером фотона, рост размеров сгустка при множественных столкновениях означает удлинение волны. Наблюдатель воспринимает это как \emph{красное смещение} \cite{hubble1929}. Этот механизм, объясняющий красное смещение как статистический процесс \emph{старения фотона}, не требует расширения пространства или гипотезы о тёмной энергии. Следует отметить, что данная концепция относится к классу моделей, исторически известных как гипотезы ``усталости света''. Однако, в отличие от ранних \textit{ad hoc} моделей, которые потерпели неудачу при столкновении с ключевыми наблюдательными тестами --- такими как отсутствие предсказываемого ими эффекта растяжения времени (time dilation) в кривых блеска сверхновых типа Ia \cite{goldhaber2001-dilation} --- ``старение фотона'' в пасталогии является прямым дедуктивным следствием механики столкновений пастонов. Это открывает возможность для построения более сложной модели, которая потенциально способна согласоваться с наблюдательными данными, где простые гипотезы потерпели крах.

\paragraph{Предельное старение.}
В предельном случае фотон не распадается на некогерентный поток, а постепенно размывается в облако пастонов. При этом ключевым фактором является не столько рост его линейных размеров, сколько падение локальной плотности пастонов внутри сгустка. Когда плотность становится слишком низкой, облако теряет способность передавать дискретный импульс, необходимый для взаимодействия с веществом, и фактически перестаёт обладать свойствами фотона. На космологических временных масштабах именно такая форма предельного старения может быть связана с феноменом \emph{реликтового излучения}, которое в пасталогии трактуется как статистический фон рассеянных пастонов, утративших способность существовать в форме фотонов. На космологических масштабах наблюдаемое реликтовое излучение является статистическим результатом предельного старения множества фотонов.

\subsubsection*{Реликтовое излучение}\label{subsubsec:relic-radiation}
В стандартной космологии реликтовое излучение трактуется как «эхо» единственного Большого Взрыва \cite{alpher1948-bigbang, penzias1965-cmb} и рассматривается как свидетельство горячего начала Вселенной.
В пасталогии же оно получает иное объяснение: реликтовое излучение — это не память о сингулярном событии, а тепловой фон, возникающий в результате многократного рассеяния и постепенного старения света при его взаимодействии с пастонами.
Такой процесс носит универсальный характер и происходит повсеместно, обеспечивая равномерность наблюдаемого фона без необходимости постулирования уникального акта творения.
Таким образом, реликтовое излучение в пасталогии отражает не начало, а непрерывную динамику взаимодействия света с фундаментальной средой пастонов.

\subsection{Разрешение космологических загадок}\label{subsec:cosmological-solutions}

Пасталогия, благодаря своим фундаментальным принципам и механизмам, предлагает уникальные и последовательные объяснения для ряда космологических загадок, которые представляют собой серьёзные вызовы для стандартной модели.

\subsubsection{Большой Взрыв как рядовое событие}\label{subsubsec:big-bang}

Следует отметить, что и реликтовое излучение, обычно интерпретируемое как остаточное тепло единственного Большого Взрыва, в пасталогии понимается как естественный результат постоянных процессов рассеяния и потому не может рассматриваться доказательством сингулярного происхождения Вселенной.
В стандартной космологии концепция Большого Взрыва описывает начало Вселенной из сингулярности, предполагая единое событие, с которого началось расширение пространства и возникла вся материя. Пасталогия же, отвергая релятивистские теории и их постулаты, указывает на отсутствие необходимости в сингулярности для объяснения происхождения крупномасштабных структур. Вместо того чтобы тратить время и силы на просчитывание и достраивание ``костылей'' к устаревшей теории, пасталогия предлагает иное видение: ``Большой Взрыв'' интерпретируется как локальное, но колоссальное и рядовое явление, инициированное Невидимым Летающим Макаронным Монстром как часть его циклического творения мироздания. Хотя сама идея циклической Вселенной рассматривалась и в других космологических моделях \cite{steinhardt2007-cyclic}, пасталогия предлагает уникальный механистический сценарий, основанный на столкновении гигантских агрегатов материи, постоянно происходящего в различных уголках бесконечной Вселенной.
Этот процесс является частью непрерывного цикла материи, состоящей из вечных и неразрушимых пастонов. Суть механизма Большого Взрыва в пасталогии заключается в следующем:
\begin{enumerate}
    \item Гравитационная конденсация и формирование тефтелей: В вечной и бесконечной Вселенной материя (пастоны) постепенно попадает в гравитационные ловушки. Это приводит к её конденсации в различные объекты --- от мелких до крупных, которые в конечном итоге поглощаются чёрными дырами. Эти чёрные дыры, достигая гигантских размеров и созревая для участия в акте творения, выделяются в особую категорию и называются в пасталогии тефтелями. Поскольку время для них не ограничено, тефтели продолжают ``отъедаться'', поглощая всё, до чего может дотянуться их гравитация.
    \item Движение к столкновению: В результате этого процесса могут возникать гигантские пустоты, в которых остаются лишь немногие, но исключительно массивные тефтели, движущиеся равномерно и прямолинейно. На огромных космических расстояниях эти гигантские объекты, преодолевая свою колоссальную инерцию, начинают очень медленно двигаться друг к другу. Постепенно скорость их сближения и точность направления друг на друга увеличиваются, превращая их в самонаводящиеся ``снаряды''.
    \item Локальный Большой Взрыв как акт творения: Рано или поздно происходит лобовое (или близкое к нему) столкновение двух таких гигантских тефтелей. Это событие является кульминацией процесса, инициированного ЛММ, и представляет собой акт творения новой области Вселенной. Несмотря на свою локальность в масштабах бесконечной Вселенной, оно является колоссальным и энергетически насыщенным. Удар порождает мощнейший разлёт материи во все стороны. Этот разлёт неравномерен из-за разницы масс сталкивающихся тефтелей и неточности их взаимного наведения.
    \item Формирование наблюдаемой картины Вселенной:
    \begin{itemize}
        \item Разлёт мелких частиц: Больше всего разлетается самых мелких частиц --- фоновых пастонов. Именно такие столкновения во всех уголках Вселенной постоянно пополняют фон пространства этими вездесущими пастонами.
        \item Образование новых объектов: Меньше всего разлетается крупных, плотных кусков материи, которые сразу же начинают оформляться в сгустки. Среди них формируются новые чёрные дыры всех размеров, хотя и многократно меньших, чем столкнувшиеся гиганты. Из этих разлетевшихся пастонов и их агрегаций со временем формируются звёзды, галактики и крупномасштабные структуры, которые мы наблюдаем. Картина локальной области Вселенной после такого столкновения будет выглядеть именно так, как мы наблюдаем в реальном мире, с присущими ей неравномерностями и структурами. Эти неравномерности в крупномасштабной картине наблюдаемой части Вселенной потенциально могут быть использованы для реконструкции столкновений тефтелей, лежащих в основе их происхождения.
    \end{itemize}
\end{enumerate}
Таким образом, Большой Взрыв в пасталогии --- это не уникальное начало, а рядовое, циклическое событие перерождения материи, происходящее в вечной и бесконечной Вселенной. Он обеспечивает непрерывное формирование и обновление видимых областей мироздания, объясняя наблюдаемое движение галактик как следствие начального импульса от таких столкновений, а не расширения самого пространства.

\subsubsection{Природа ``тёмного вещества''}\label{subsubsec:dark-matter}

В стандартной космологии ``тёмное вещество'' постулируется для объяснения таких явлений, как аномальные скорости вращения галактик \cite{rubin1980}, --- наблюдательной загадки, которая также привела к разработке альтернативных теорий, например, Модифицированной ньютоновской динамики (MOND) \cite{milgrom1983-mond}. В пасталогии необходимость в обеих этих гипотезах исчезает.
``Тёмное вещество'' в пасталогии --- это гигантские, разрежённые скопления особых видов пастонов, называемых тефтонами. Тефтон --- это пастон, который отличается от фоновых пастонов своими размерами: тефтон имеет размер, достаточный для того, чтобы на него оказывалось гравитационное давление и он сам обладает способностью к экранированию (гравитационной тенью). Фоновый пастон, напротив, имеет недостаточный для этого размер и является лишь переносчиком импульса, не подверженным заметному гравитационному воздействию. Тефтоны, как и фоновые пастоны, движутся равномерно и прямолинейно от удара до удара. Вероятно, именно из тефтонов могут состоять стабильные ``фундаментальные'' частицы, известные в стандартной физике.
Эти тефтоны образуют огромные, разрежённые гало вокруг галактик. Такое гало из тефтонов обладает значительной суммарной гравитационной тенью и может рассматриваться как гигантский космический объект с очень низкой плотностью. Его упорядоченность ограничивается лишь общей геометрией этого гало.
Воздействие на тела, находящиеся внутри такого гало тёмного вещества, называется гравитационной вязкостью. Этот эффект представляет собой прямое гравитационное воздействие, возникающее из распределения гравитационных теней от тефтонов, составляющих гало. Если тело находится в центре гало, гравитационное воздействие на него будет равномерным со всех сторон из-за симметричного распределения гравитационных теней от окружающих тефтонов. Однако, чем ближе тело будет находиться к краю гало, тем сильнее будет проявляться гравитационное воздействие, поскольку дисбаланс гравитационных теней станет более выраженным, и суммарный вектор этого воздействия будет направлен к центру гало. Именно этот градиент гравитационного давления и обуславливает ``вязкость''.
Таким образом, гало из тефтонов оказывает прямое гравитационное воздействие на обычную материю, но при этом остаётся невидимым для любых форм электромагнитного излучения. Это объясняется тем, что, будучи пастонами, тефтоны не имеют внутренней структуры и не способны к сложным эмерджентным взаимодействиям, таким как излучение или поглощение света. Это объясняет, почему, согласно современным наблюдательным данным, ``тёмное вещество'' проявляет себя только через гравитационное воздействие и до сих пор не было обнаружено в экспериментах по прямому детектированию \cite{bertone2005-review}.

\subsubsection{Отсутствие ``тёмной энергии''}\label{subsubsec:no-dark-energy}

Концепция ``тёмной энергии'' была введена для объяснения кажущегося ускоренного расширения Вселенной \cite{riess1998}, которое следует из интерпретации дополнительного красного смещения как расширения пространства. Поскольку в пасталогии дополнительное красное смещение объясняется феноменом ``старения света'' (см.~\ref{subsec:light-aging}), заключающимся в увеличении длины волны фотона из-за механизма ``рокировки'' при многократных столкновениях с фоновыми пастонами, ведущих к увеличению линейных размеров фотона, необходимость в существовании ``тёмной энергии'' полностью отпадает.

Вселенная в пасталогии является статичной и бесконечной, она не расширяется и не имеет центра. Наблюдаемое красное смещение является следствием увеличения линейных размеров фотонов при прохождении больших расстояний, а не удалением галактик друг от друга или ускоренным расширением пространства. Таким образом, пасталогия предлагает естественное разрешение проблемы ``тёмной энергии'', устраняя потребность в гипотетической сущности, вызывающей ``расширение Ничто''.

\subsection{Корпускулярно-волновой дуализм}\label{subsec:wave-particle-duality}

Явление корпускулярно-волнового дуализма, которое в квантовой механике описывает частицы как обладающие одновременно свойствами и частиц, и волн, в пасталогии получает иное, более механистическое объяснение. Здесь нет необходимости постулировать двойственную природу материи или ``коллапс волновой функции'', который является одним из наиболее загадочных и спорных аспектов квантовой механики.

Следует отметить, что данный подход, разделяющий частицу и ведущую её волну, имеет концептуальное сходство с теориями ``пилотной волны'', такими как механика де Бройля — Бома \cite{bohm1952-interpretation}. Однако, в отличие от этих теорий, которые постулируют существование абстрактной, нелокальной ``квантовой потенциальной'' волны как отдельной фундаментальной сущности, пасталогия предлагает чисто механистическую и локальную картину. В её рамках ``волна'' не является независимым полем, а представляет собой эмерджентное, физическое возмущение в среде вездесущих фоновых пастонов, генерируемое движением самой частицы.

В пасталогии:
\begin{enumerate}
    \item \textbf{``Частица'' как сгусток пастонов.} Фундаментальные ``частицы'' (такие как электрон, протон, а также более сложные образования) представляют собой стабильные сгустки пастонов (см. раздел~\ref{subsec:light-aging}). Эти сгустки обладают определённой инертной массой (как совокупность масс составляющих их пастонов) и движутся в пространстве, взаимодействуя с другими пастонами только путём абсолютно упругих столкновений. Их дискретная, локализованная природа соответствует ``корпускулярной'' части дуализма.
    \item \textbf{``Волновое'' поведение как эмерджентный эффект.} Волновые свойства, наблюдаемые у таких ``частиц'', являются не внутренним свойством самой частицы, а эмерджентным эффектом, возникающим из её взаимодействия с фоном вездесущих пастонов. Когда сгусток пастонов (``частица'') движется через пространство, он постоянно сталкивается с множеством фоновых пастонов, движущихся во всех направлениях. Эти столкновения генерируют регулярный волновой фронт из фоновых пастонов, который распространяется вокруг движущейся частицы. Этот волновой фронт является не ``волной вероятности'', а реальным, физическим возмущением в среде фоновых пастонов, распространяющимся в соответствии с законами механики.
    \item \textbf{Длина волны и взаимодействие со средой.} Фоновый волновой фронт имеет определённую длину волны, которую мы отождествляем с длиной волны создающей его частицы. По этой причине при прохождении через среду (например, твёрдые тела), волна может рассеиваться, если её длина сопоставима с расстоянием между частицами среды. Однако, проходя через щели достаточного размера, волновой фронт остаётся волной, поскольку его масштабы существенно больше размеров элементов преграды.
    \item \textbf{Интерференция и дифракция.} При прохождении ``частицы'' через препятствия, такие как щели (в двухщелевом эксперименте, см.~\ref{subsec:slit-experiments}), этот фоновый волновой фронт проходит через все доступные щели и интерферирует, создавая характерную интерференционную картину в среде фоновых пастонов за препятствием.
    \item \textbf{Детектирование.} Сама ``частица'' (дискретный сгусток пастонов) продолжает своё движение и в конечном итоге достигает детектора. Однако её траектория не является произвольной; она модулируется максимумами интерферирующего волнового фронта фоновых пастонов. То есть, вероятность обнаружения частицы в той или иной точке детектора определяется интенсивностью волнового фронта фоновых пастонов в этой точке. Это объясняет, почему, хотя частица и является дискретным объектом, её попадания на детекторе статистически распределяются в соответствии с волновой картиной. Детектирование всегда регистрирует дискретный сгусток пастонов, а не ``волну''.
\end{enumerate}

Таким образом, корпускулярно-волновой дуализм в пасталогии не является мистическим внутренним свойством частиц, а логически выводится из их природы как сгустков пастонов и их механического взаимодействия с фоновой средой. ``Частица'' всегда остаётся частицей, а ``волна'' --- это реальное, физическое возмущение в среде фоновых пастонов, которое влияет на поведение частицы.

\subsection{Другие свойства частиц (электрический заряд, спин, цвет)}\label{subsec:particle-properties}

В дополнение к инертной массе и волновым свойствам, проявляемым через взаимодействие с фоновыми пастонами, наблюдаемые фундаментальные частицы обладают рядом других характеристик, таких как электрический заряд, спин и ``цвет'' (в рамках квантовой хромодинамики) \cite{pdg2024}. В пасталогии эти свойства не постулируются как изначально присущие пастонам, а объясняются как высокоуровневые, эмерджентные эффекты, возникающие из специфических конфигураций, сложной внутренней динамики и пространственного расположения пастонов в организованных сгустках.

\begin{itemize}
    \item \textbf{Электрический заряд.} Предполагается, что электрический заряд является результатом определённой устойчивой, но динамичной асимметрии в распределении и движении пастонов внутри частицы, создающей специфические паттерны в локальном давлении фоновых пастонов или в их обмене импульсом с другими заряженными частицами (см.~\ref{subsec:gravity-pressure}). Это может быть связано с особенностями внутренней циркуляции пастонов или их геометрическим расположением, которые приводят к давлению, имеющему направленный характер.
    \item \textbf{Спин.} Спин частицы, являющийся фундаментальной характеристикой, описывается в пасталогии как результат вращения пастонов внутри сгустка. Это внутреннее движение создаёт определённый гироскопический эффект и влияет на то, как частица взаимодействует с фоновой средой и другими частицами, передавая импульс вращения.
    \item \textbf{``Цвет'' (сильное и слабое взаимодействия).} Сильные и слабые взаимодействия, включая свойства, известные как ``цвет'' в стандартной модели, имеют ту же фундаментальную природу, что и гравитация --- они являются проявлением обмена импульсом через столкновения пастонов, создающего эффекты давления. Однако, в отличие от гравитации, эти взаимодействия проявляются с иным характером и интенсивностью на малых масштабах из-за специфических конфигураций и внутренней динамики пастонов, образующих частицы. Результаты компьютерного 3D-моделирования процессов убедительно демонстрируют эти механизмы, но обзорный формат статьи не подходит для описания этих экспериментов.
\end{itemize}

Следует признать, что детальное выведение и математическое описание этих сложных свойств из фундаментального допущения о пастонах требует дальнейшего глубокого развития теории. На данном этапе пасталогия предлагает качественную основу для понимания этих явлений как следствий сложных механических взаимодействий и конфигураций простейших элементов, а не как изначально присущих фундаментальных полей или невыводимых квантовых чисел.

\section{Предсказания и фальсифицируемость пасталогии}\label{sec:predictions}

Одной из важнейших черт любой научной теории является её способность делать проверяемые предсказания, которые могут быть экспериментально или наблюдательно подтверждены, либо, напротив, опровергнуты. Только наличие таких предсказаний позволяет отличить научную теорию от метафизической концепции \cite{popper1959}. Пасталогия, несмотря на свою радикально иную фундаментальную природу, не только объясняет существующие явления, но и делает ряд уникальных предсказаний, которые отличаются от таковых в стандартной физической модели. Эти расхождения предлагают пути для потенциальной фальсификации теории.

\subsection{Неэквивалентность инертной и ``гравитационной'' масс для сверхплотных объектов}\label{subsec:mass-inequivalence}

Как было подробно описано (см.~\ref{subsec:mass-nonequivalence}), в пасталогии гравитационный эффект не тождественен инертной массе для сверхплотных объектов. Это является прямым следствием механизма ``взаимного экранирования пастонов'' внутри таких тел.

\paragraph{Предсказание.} Согласно пасталогии, для объектов с чрезвычайно высокой плотностью (например, нейтронных звёзд, чёрных дыр), их эффективность экранирования (``гравитационная масса'' в традиционной терминологии) будет меньше их инертной массы. Чем выше плотность и больше линейные размеры сверхплотного объекта, тем сильнее должен проявляться этот эффект.

\paragraph{Способы фальсификации/проверки.} Проверка этого предсказания требует прецизионных астрофизических наблюдений за системами, включающими сверхплотные объекты. Если бы удалось провести точные измерения гравитационного воздействия такого объекта (например, через его влияние на орбиты других тел в двойных или кратных системах, или через анализ движения газа вокруг него) и сравнить его с его инертной массой (которая может быть независимо определена из динамических параметров системы, таких как период обращения и скорости), то обнаружение точного соответствия между ними для всех плотностей, вплоть до самых высоких, стало бы прямым опровержением предсказаний пасталогии.

И наоборот, если бы были обнаружены даже небольшие отклонения в сторону уменьшения гравитационного эффекта по сравнению с инертной массой для сверхплотных объектов, это стало бы сильным подтверждением предсказаний пасталогии. Текущие наблюдательные данные могут быть недостаточны для выявления столь тонких расхождений из-за ограничений в точности измерений или необходимости длительных наблюдений. Однако развитие астрофизических инструментов и методик, таких как более точные измерения орбитальных параметров двойных пульсаров, аккреционных дисков вокруг чёрных дыр или ``гравитационно-волновых'' сигналов от систем перед слиянием, потенциально может обеспечить необходимую точность для такой проверки в будущем.

\subsection{Предсказания для Астрономии Гравитационных Возмущений}\label{subsec:gravitational-astronomy}

Представление гравитации в пасталогии как эмерджентного эффекта давления фоновых пастонов (см.~\ref{subsec:gravity-pressure}) приводит к конкретным, проверяемым предсказаниям, отличающимся от интерпретации регистрируемых явлений в Общей Теории Относительности. Если в стандартной физике наблюдаемые колебания считаются ``гравитационными волнами'' --- возмущениями пространственно-временной ткани (см.~\ref{subsec:gravitational-waves}), то в пасталогии эти регистрируемые ``гравитационно-волновые'' сигналы от сверхплотных объектов являются периодическими изменениями совокупной гравитационной тени, создаваемыми динамическими системами из сверхплотных объектов.

\paragraph{Предсказания.}
\begin{enumerate}
    \item \textbf{Зависимость интенсивности регистрируемых сигналов от ракурса наблюдения.} В пасталогии эти наблюдаемые сигналы возникают из-за изменения взаимного экранирования сверхплотных объектов в двойных или кратных системах при их вращении. Соответственно, интенсивность регистрируемого эффекта должна сильно зависеть от ракурса наблюдения. Максимальный сигнал будет фиксироваться при наблюдении сверхплотных двойных систем ``с ребра'' (то есть из плоскости их орбиты), где эффект взаимного затенения и его изменения наиболее выражены. При наблюдении ``с полюса'' (перпендикулярно плоскости орбиты) или под другими углами интенсивность сигнала будет значительно ниже, вплоть до полного отсутствия или крайне слабого проявления, так как эффект затенения будет минимальным или нулевым. Это отличается от ОТО, где гравитационные волны распространяются во все стороны и их интенсивность имеет определённую диаграмму направленности \cite{maggiore2007-gw}, но не обусловлена прямым затенением.
    \item \textbf{``Предвестники слияния'' и точное прогнозирование.} Предвестники возможны для систем, наблюдаемых ``с ребра'' или близко к такому расположению. Важно, чтобы происходило прохождение сверхплотных тел на фоне друг друга. На поздних стадиях сближения двух тел в двойной системе, когда начинается существенное геометрическое перекрытие их гравитационных теней, пасталогия предсказывает возможность регистрации специфических ``предвестников слияния''. Вместо непрерывно нарастающего квазипериодического сигнала, пасталогия предполагает появление ``единичных'' малых колебаний или ``толчков'' в гравитационном давлении. Эти колебания будут регистрироваться со значительными, но постепенно укорачивающимися временными промежутками между ними, по мере того как тела двойной системы всё сильнее сближаются и их относительная скорость нарастает. Толчки со временем должны становиться короче и мощнее. Следует тщательно проверить существующие записи регистрации событий на наличие таких одиночных толчков. Они могут быть чрезвычайно редкими и слабыми, при этом иметь большую длительность и медленные нарастание и откат, что объясняет их текущую необнаруженность, но их наличие является ключевым подтверждением пасталогии. Если удастся обнаружить толчки от разных систем, они могут на первый взгляд быть распределены во времени хаотично, однако их детальный анализ позволит разделить их по объектам и с высокой точностью предсказать следующие толчки от каждой конкретной системы. Такая картина позволит:
    \begin{itemize}
        \item Точно предсказывать время слияния: анализируя динамику укорочения интервалов между ``толчками'', можно будет с высокой точностью рассчитать оставшееся время до кульминации события.
        \item Прогнозировать силу будущих всплесков: характеристики этих предвестников могут нести информацию о массах и скоростях сближающихся тел, позволяя прогнозировать мощность основного всплеска гравитационного давления.
        \item Целенаправленное наблюдение: эти предвестники позволят астрономам заблаговременно наводить телескопы и детекторы, чтобы целенаправленно наблюдать события слияния, а не обнаруживать их задним числом по уже произошедшему сигналу, как это часто происходит с текущими регистрациями слияний.
    \end{itemize}
\end{enumerate}

Эти предсказания пасталогии открывают новые возможности для астрофизических наблюдений и потенциально могут быть проверены с помощью высокочувствительных детекторов, предназначенных для регистрации подобных гравитационных возмущений.

\subsection{Предсказания для модифицированных опытов со щелями}\label{subsec:slit-experiments}

Объяснение корпускулярно-волнового дуализма в пасталогии (см.~\ref{subsec:wave-particle-duality}) как взаимодействия дискретного сгустка пастонов (частицы) с реальным, физическим волновым фронтом в среде фоновых пастонов приводит к конкретному, проверяемому предсказанию для модифицированного двухщелевого эксперимента, которое существенно отличается от интерпретаций стандартной квантовой механики.

\paragraph{Ключевые положения пасталогии.}
\begin{itemize}
    \item Частица --- это всегда дискретный, локализованный сгусток пастонов.
    \item ``Волна'' --- это реальное возмущение в вездесущем фоне пастонов, генерируемое движением частицы. Этот волновой фронт проходит через все доступные щели и интерферирует, при этом он сам по себе не регистрируется детекторами и интерферирует с другими фронтами, не разрушая их.
    \item Частица, проходящая через одну из щелей, ``направляется'' к максимумам интерференционной картины, формируемой фоновым волновым фронтом.
\end{itemize}

\paragraph{Предсказание.} Рассмотрим модифицированный двухщелевой эксперимент, в котором источник частиц (например, фотонов или электронов \cite{tonomura1989-electron}) настроен так, что все частицы заведомо проходят исключительно через одну из двух щелей. Это можно обеспечить, попеременно открывая то одну, то другую щель и фиксируя прохождение частиц только через активную щель. Таким образом, каждая частица проходит только через одну щель и не может ``интерферировать сама с собой'' или с другими частицами, пролетающими через вторую щель, как предполагается в стандартной квантовой механике. В пасталогии, однако, фоновый волновой фронт, генерируемый движением частицы, беспрепятственно проходит через обе щели и интерферирует, создавая характерную интерференционную картину за ними. Поскольку частица ``направляется'' этим волновым фронтом, пасталогия предсказывает, что при гарантированном прохождении всех частиц только через одну щель на детекторе всё равно будет наблюдаться полноценная интерференционная картина двух щелей с центром напротив активной щели.

\paragraph{Предсказание стандартной квантовой механики.} В стандартной квантовой механике состояние частицы в двухщелевом эксперименте описывается как суперпозиция путей через обе щели. Интерференционная картина возникает благодаря взаимодействию этих путей. Однако, если путь частицы однозначно определён (например, частица проходит только через одну щель, а вторая щель не вносит вклада в волновую функцию), интерференционные эффекты исчезают. В результате на детекторе наблюдается дифракционная картина одной щели --- одна яркая полоса, соответствующая проекции открытой щели, без характерных интерференционных максимумов и минимумов.

Формально это выражается так. В стандартной квантовой механике амплитуда состояния в области экрана задаётся суммой двух компонент:
\begin{equation}
    \Psi(x) = \Psi_1(x) + \Psi_2(x).
\end{equation}
Тогда вероятность регистрации частицы равна
\begin{equation}
    P(x) = |\Psi(x)|^2 = |\Psi_1(x)|^2 + |\Psi_2(x)|^2 + \Psi_1^*(x)\Psi_2(x) + \Psi_2^*(x)\Psi_1(x),
\end{equation}
где последние два слагаемых описывают интерференционные члены.

В модифицированном эксперименте при гарантированном прохождении частиц только через одну щель имеем
\begin{equation}
    \Psi_2(x) = 0 \Rightarrow \Psi(x) = \Psi_1(x), \quad P(x) = |\Psi_1(x)|^2.
\end{equation}

Таким образом, стандартная квантовая механика предсказывает \emph{дифракционную картину одной щели} без интерференции.

\paragraph{Сравнение и фальсифицируемость.} Таким образом, предсказания пасталогии и стандартной квантовой механики принципиально расходятся:
\begin{itemize}
    \item Стандартная квантовая механика: при прохождении частиц только через одну щель наблюдается дифракционная картина одной щели.
    \item Пасталогия: при прохождении частиц только через одну щель наблюдается интерференционная картина двух щелей.
\end{itemize}
Этот модифицированный двухщелевой эксперимент является фальсифицирующим тестом, позволяющим однозначно различить предсказания пасталогии и стандартной квантовой механики. Наблюдение интерференционной картины в условиях гарантированного однощелевого прохождения частиц стало бы сильным подтверждением пасталогии, тогда как дифракционная картина подтвердила бы стандартную квантовую механику.

\subsection{Объяснение существующих астрономических аномалий}\label{subsec:astronomical-anomalies}

Пасталогия предлагает проверяемые предсказания для астрономических наблюдений, которые в стандартной космологии воспринимаются как аномалии, требующие дополнительных гипотез. К таким наблюдениям относятся существование чёрных дыр с массами, превышающими ожидаемые, объекты с экстремально высоким красным смещением, длительные вспышки, такие как ``Страшная Барби'', и объекты с аномально низким красным смещением. Эти явления естественно объясняются в рамках пасталогии как следствия локальных Больших Взрывов и поддаются наблюдательной проверке.

\paragraph{Предсказания.}
\begin{enumerate}
    \item \textbf{Чёрные дыры с аномально большими массами в наблюдаемой Вселенной.} В стандартной космологии формирование сверхмассивных чёрных дыр на больших красных смещениях (\( z > 6 \)) проблематично \cite{larson2023-jwst-bh} --- это расхождение усугубляется недавними открытиями популяций ранних чёрных дыр телескопом JWST \cite{maiolino2024-jades-bh}. Пасталогия предсказывает, что такие чёрные дыры --- прямые осколки локального Большого Взрыва (см. раздел~\ref{subsubsec:big-bang}), столкновения двух гигантских тефтелей (созревших чёрных дыр), инициирующего наблюдаемую часть Вселенной, когда ЛММ ``ударил тефтелем о тефтель''. Пасталогия предсказывает, что такие чёрные дыры будут сопровождаться другими плотными структурами (например, компактными звёздными скоплениями) с аномально высокой плотностью в их окрестностях.
    \item \textbf{Зрелые галактики с критически большими красными смещениями.} Наблюдение зрелых галактик на \( z > 10 \) \cite{labbe2023-jwst-galaxies} и даже \( z \approx 14 \) \cite{carniani2024-z14} противоречит стандартной космологии \(\Lambda\)CDM, которая требует значительных корректировок для объяснения их обилия и массы \cite{lu2024-jwst-lcdm-tension}. Пасталогия предсказывает, что большинство таких галактик демонстрируют высокую металличность и сложную морфологию, обусловленные формированием из материала локального Большого Взрыва. Красное смещение интерпретируется как комбинация эффекта Доплера и старения света (см. раздел~\ref{subsec:light-aging}), а бесконечная и вечная Вселенная устраняет временные ограничения на формирование структур.
    \item \textbf{Длительные неослабевающие вспышки от ``Малых Взрывов''.} Пасталогия предсказывает, что лобовые столкновения рядовых чёрных дыр в крупных войдах, где низкая плотность объектов позволяет прямолинейное сближение без орбитального танца, порождают ``Малые Взрывы''. Эти события проявляются как длительные (несколько лет), неослабевающие вспышки, постепенно расходящиеся в пространстве, без обнаруживаемой галактики-хозяина. Потенциальным примером является ``Страшная Барби'' (AT 2021lwx) \cite{subrayan2023-barbie}, с энергией \( 1.5 × 10^{53} \) эрг и длительностью \( >3 \) лет, которая может быть результатом такого столкновения, а не стандартного события приливного разрушения.
    \item \textbf{Объекты с аномально низким красным смещением.} Пасталогия предсказывает, что в некоторых направлениях могут быть обнаружены редкие космические тела с красным смещением, неестественно низким для их расстояния (оценённого, например, по светимости или угловому размеру) \cite{arp1987-quasars}, как следствие другого локального Большого Взрыва в бесконечной Вселенной (см. раздел~\ref{subsubsec:big-bang}). Это обусловлено частичной компенсацией красного смещения старения фотонов (см. раздел~\ref{subsec:light-aging}) синим доплеровским смещением от движения навстречу Земле.
\end{enumerate}
 
\paragraph{Способы проверки и фальсификации.} Проверка предсказаний пасталогии возможна с помощью телескопов, таких как JWST или ZTF. Для чёрных дыр (пункт 1) наблюдения должны выявить корреляцию их местоположения с плотными структурами в окрестностях, указывающими на происхождение от локального Большого Взрыва. Если чёрные дыры распределены равномерно без таких корреляций, это ослабит пасталогию. Для галактик на \( z > 10 \) (пункт 2) спектроскопия должна показать, что большинство из них имеют высокую металличность и сложную структуру. Если большинство галактик окажутся простыми и с низкой металличностью, это ослабит пасталогию. Для ``Малых Взрывов'' (пункт 3) наблюдения в войдах должны выявить длительные вспышки без галактики-хозяина, такие как ``Страшная Барби''. Если такие вспышки ослабевают, как типичные события приливного разрушения или гамма-всплески, это ослабит пасталогию. Для объектов с аномально низким красным смещением (пункт 4) спектроскопия должна выявить редкие объекты с красным смещением, меньшим, чем ожидается для их расстояния. Если такие объекты не будут обнаружены в течение длительного времени, это ослабит пасталогию, учитывая их редкость. Обнаружение предсказанных характеристик стало бы сильным подтверждением пасталогии.

\section{Обсуждение и Заключение}\label{sec:discussion-conclusion}

\subsection{Сильные стороны Пасталогии}\label{subsec:strengths}

Пасталогия, как новая аксиоматическая теория мироздания, обладает рядом фундаментальных преимуществ, которые выгодно отличают её от существующих научных парадигм и придают ей особую ценность в поиске полного и логически обоснованного описания Вселенной:

\begin{itemize}
    \item \textbf{Логическая экономичность:} Одним из наиболее выдающихся достоинств пасталогии является её способность построить целостную и всеобъемлющую картину мира, исходя из единственного фундаментального аксиоматического допущения: материя делима, но не до бесконечности. Эта предельная простота исходных принципов делает пасталогию образцом логической экономичности в научном построении.
    
    \item \textbf{Внутренняя непротиворечивость и целостность:} Все положения пасталогии, начиная от природы фундаментальных частиц (пастонов) и заканчивая объяснением космологических явлений, дедуктивно выводятся из этого единственного аксиоматического принципа. Это обеспечивает исключительную внутреннюю непротиворечивость теории и её концептуальную целостность, где каждая часть логически связана с другими.
    
    \item \textbf{Элегантность объяснений:} Пасталогия предлагает элегантные и интуитивно понятные объяснения сложных физических явлений, сводя их к простым механическим принципам взаимодействия пастонов. Гравитация как давление, свет как сгустки пастонов, а корпускулярно-волновой дуализм как взаимодействие частицы с физическим волновым фронтом в среде --- все эти концепции просты, наглядны и не требуют введения абстрактных математических конструктов без физического аналога.
    
    \item \textbf{Устранение ``сверхъестественных'' или необоснованных сущностей:} Одним из ключевых преимуществ пасталогии является её способность устранять из фундаментального описания мироздания ряд сущностей, которые в стандартной физике часто кажутся ``сверхъестественными'' или вводятся ad hoc для объяснения наблюдений. Пасталогия обходится без фундаментальных полей (заменяя их механическим взаимодействием пастонов), без ``тёмной энергии'' (поскольку не постулирует ускоренного расширения пространства) и без неинтуитивного ``коллапса волновой функции'' (объясняя его как нарушение реального волнового фронта).
    
    \item \textbf{Высокая объяснительная и предсказательная сила:} Несмотря на свою аксиоматическую простоту, пасталогия демонстрирует высокую объяснительную силу для широкого круга наблюдаемых макроскопических и микроскопических явлений, включая гравитацию, свет, корпускулярно-волновой дуализм, а также астрономические аномалии. Более того, она предлагает ряд конкретных, проверяемых предсказаний, отличающихся от предсказаний стандартной физики (например, касательно нетождественности гравитационной и инертной масс для сверхплотных объектов, специфики гравитационных возмущений и поведения частиц в модифицированных опытах со щелями), что открывает путь для её эмпирической проверки.
\end{itemize}

\subsection{Место пасталогии в науке}\label{subsec:place-in-science}

Пасталогия занимает уникальное положение в ландшафте современного научного знания, поскольку она существенно отличается от традиционного эмпирического подхода, характерного для физики. По своей сути, пасталогия --- это не физика в её эмпирическом понимании, которая строит свои теории, отталкиваясь от наблюдений и экспериментов. Напротив, пасталогия является фундаментальной аксиоматической наукой, цель которой --- дедуктивное построение мироздания из фундаментальных основ. Она начинает с единственного, предельно простого аксиоматического допущения и через строгую дедукцию стремится вывести все известные физические явления и космологические структуры.

Её задача --- не столько описывать уже наблюдаемое (хотя она это успешно делает), сколько предоставить логически непротиворечивую и самодостаточную картину реальности, вытекающую из минимального набора постулатов. Такой подход сближает пасталогию скорее с математическими системами или философскими основаниями науки, предлагая альтернативный путь к пониманию фундаментальной природы Вселенной.

Тем не менее, пасталогия не стремится изолироваться от эмпирической науки. Напротив, её сила заключается в потенциальной ``встрече'' с физикой. Эта ``встреча'' произойдет, когда эмпирические данные, полученные физикой, и логические выводы, сделанные пасталогией, сойдутся и полностью совпадут. Если предсказания пасталогии будут подтверждены экспериментально, это станет моментом, когда аксиоматическое построение и эмпирическое наблюдение сольются, формируя единую, полную и глубоко обоснованную картину мироздания. В таком случае пасталогия сможет предоставить физике тот фундаментальный уровень понимания, который до сих пор оставался за пределами её досягаемости, предлагая не просто описание, а объяснение причин всех наблюдаемых явлений.

\subsection{Будущие направления исследований}\label{subsec:future-directions}

Несмотря на то, что пасталогия уже предлагает цельную и внутренне непротиворечивую картину мироздания, а также объясняет ряд ключевых явлений и аномалий, её полное развитие требует значительных усилий в нескольких ключевых направлениях. Эти направления представляют собой дорожную карту для будущих теоретических и, возможно, экспериментальных исследований:

\begin{itemize}
    \item \textbf{Разработка формального математического аппарата:} На данном этапе пасталогия представлена преимущественно на концептуальном и качественном уровне, демонстрируя логическую стройность своих принципов. Однако для перехода к количественным предсказаниям и строгому научному анализу критически важна разработка полноценного формального математического аппарата. Это включает в себя создание математических моделей, описывающих динамику отдельных пастонов, их столкновения, формирование стабильных агрегатов, а также распространение возмущений в среде фоновых пастонов. Такой аппарат позволит вычислять точные значения наблюдаемых величин и проводить симуляции сложных явлений.
    
    \item \textbf{Детальное теоретическое моделирование формирования стабильных сложных частиц и выведение их свойств:} Существующие положения пасталогии объясняют природу ``частиц'' как сгустков пастонов. Следующим этапом является детальное теоретическое моделирование и выведение свойств конкретных стабильных сложных частиц, таких как электроны, кварки, нейтрино, из поведения и конфигураций пастонов. Это включает в себя не только их массу, но и такие фундаментальные характеристики, как электрический заряд (как специфическая асимметрия в движении пастонов), спин (как внутреннее вращение агрегата пастонов) и ``цвет'' (как особая конфигурация, определяющая сильные взаимодействия). Успешное выведение этих свойств из первичных принципов пасталогии стало бы колоссальным прорывом.
    
    \item \textbf{Предложение конкретных экспериментальных установок для проверки ключевых предсказаний:} Хотя статья уже содержит ряд предсказаний, отличающихся от стандартной физики (например, касательно гравитационного влияния сверхплотных объектов и поведения частиц в модифицированных опытах со щелями), будущие исследования должны сосредоточиться на разработке конкретных, детальных проектов экспериментальных установок, способных проверить эти предсказания. Это потребует глубокого анализа технических возможностей, требований к точности измерений и разработки инновационных методик, которые смогут обнаружить тонкие эффекты, предсказываемые пасталогией, но игнорируемые или объясняемые иначе в рамках стандартной модели. Подобные эксперименты, даже если они кажутся футуристическими, являются конечной целью и критерием верифицируемости теории.
\end{itemize}

\subsection{Заключительное слово}\label{subsec:conclusion}

Настоящая статья представила аксиоматическую теорию пасталогии --- попытку переосмыслить фундаментальные основы физики и космологии с помощью минимального числа допущений. Отправляясь от единственного аксиоматического принципа --- делимости материи до мельчайших, неделимых пастонов --- пасталогия предлагает целостную и внутренне непротиворечивую картину мироздания. В её рамках такие феномены, как гравитация, свет, корпускулярно-волновой дуализм, а также ряд астрономических аномалий, получают объяснение через простые механические взаимодействия этих базовых элементов.

Пасталогия бросает вызов существующим научным парадигмам, предлагая вместо абстрактных полей и вероятностных волн конкретную физическую реальность, состоящую из дискретных частиц и их взаимодействий. Она призывает к пересмотру устоявшихся представлений о пространстве, времени, массе и энергии, предлагая заменить их на понятия, выводимые из поведения пастонов.

В конечном итоге, пасталогия выступает не просто как очередная гипотеза, а как новая парадигма в понимании Вселенной. Она открывает путь к созданию единой теории всего, основанной на логике и дедукции, и предлагает конкретные, проверяемые предсказания, которые могут стать решающим критерием её истинности. Если пасталогия выдержит проверку будущими экспериментами и наблюдениями, она способна навсегда изменить наш взгляд на Вселенную, вернув физике интуитивную ясность и глубину понимания, коренящуюся в самых элементарных принципах реальности.

\backmatter

\section*{Благодарности}\label{sec:acknowledgements}

В процессе подготовки данной статьи автор использовал ряд инструментов на основе искусственного интеллекта. Для улучшения стилистики, содействия в LaTeX-вёрстке и перевода на английский язык были применены большие языковые модели (LLM), включая несколько общедоступных систем, а также собственный инструмент автора AIS Agora. Автор тщательно проверял, редактировал и несет полную ответственность за все утверждения и окончательный текст рукописи.

\begin{thebibliography}{99}

\bibitem{einstein1916}
Einstein, A. (1916). Die Grundlage der allgemeinen Relativitätstheorie. \textit{Annalen der Physik}, 354(7), 769-822.

\bibitem{bohr1928}
Bohr, N. (1928). The Quantum Postulate and the Recent Development of Atomic Theory. \textit{Nature}, 121(3050), 580-590.

\bibitem{rubin1980}
Rubin, V. C., Ford, W. K., Jr., \& Thonnard, N. (1980). Rotational properties of 21 Sc galaxies with a large range of luminosities and radii, from NGC 4605 (R = 4kpc) to UGC 2885 (R = 122kpc). \textit{The Astrophysical Journal}, 238, 471-487.

\bibitem{riess1998}
Riess, A. G., et al. (1998). Observational evidence from supernovae for an accelerating universe and a cosmological constant. \textit{The Astronomical Journal}, 116(3), 1009-1038.

\bibitem{euclid-elements-heath}
Heath, T. L. (Trans. \& ed.). (1956). \textit{The Thirteen Books of Euclid's Elements} (2nd ed., 3 Vols.). New York: Dover Publications.

\bibitem{wiener1948}
Wiener, N. (1948). \textit{Cybernetics: Or Control and Communication in the Animal and the Machine}. New York: John Wiley \& Sons.

\bibitem{popper1959}
Popper, K. (1959). \textit{The Logic of Scientific Discovery}. London: Hutchinson.

\bibitem{kirk1983-democritus}
Kirk, G. S., Raven, J. E., \& Schofield, M. (1983). \textit{The Presocratic Philosophers: A Critical History with a Selection of Texts} (2nd ed.). Cambridge: Cambridge University Press.

\bibitem{leibniz1989-monadology}
Leibniz, G. W. (1989). The Monadology. In R. Ariew \& D. Garber (Eds. \& Trans.), \textit{Philosophical Essays} (pp. 213-225). Indianapolis: Hackett Publishing Company.

\bibitem{henderson2006}
Henderson, B. (2006). \textit{The Gospel of the Flying Spaghetti Monster}. Villard Books.

\bibitem{ockham-razor-sep}
Baker, A. (2022). Simplicity. In E. N. Zalta (Ed.), \textit{The Stanford Encyclopedia of Philosophy} (Winter 2022 ed.). Metaphysics Research Lab, Stanford University.

\bibitem{pdg2024}
Navas, S., et al. (Particle Data Group). (2024). Review of Particle Physics. \textit{Physical Review D}, 110(3), 030001.

\bibitem{feynman-lectures-v1}
Feynman, R. P., Leighton, R. B., \& Sands, M. (2011). \textit{The Feynman Lectures on Physics, Vol. 1: The New Millennium Edition}. New York: Basic Books.

\bibitem{mtw1973}
Misner, C. W., Thorne, K. S., \& Wheeler, J. A. (1973). \textit{Gravitation}. San Francisco: W. H. Freeman and Company.

\bibitem{lesage-edwards2014}
Edwards, M. R. (2014). Pushing Gravity: A Review of Le Sage-type Theories. \textit{Studies in History and Philosophy of Science Part B}, 47, 1-15.

\bibitem{einstein1920-relativity}
Einstein, A. (2015). \textit{Relativity: The Special and the General Theory}. Princeton, NJ: Princeton University Press.

\bibitem{shapiro-teukolsky1983}
Shapiro, S. L., \& Teukolsky, S. A. (1983). \textit{Black Holes, White Dwarfs, and Neutron Stars: The Physics of Compact Objects}. New York: Wiley-Interscience.

\bibitem{newton1687-principia}
Newton, I. (1999). \textit{The Principia: Mathematical Principles of Natural Philosophy} (I. B. Cohen \& A. Whitman, Trans.). Berkeley, CA: University of California Press. (Original work published 1687).

\bibitem{halliday2013-fundamentals}
Halliday, D., Resnick, R., \& Walker, J. (2013). \textit{Fundamentals of Physics} (10th ed.). Hoboken, NJ: John Wiley \& Sons.

\bibitem{newman2013-computational}
Newman, M. (2013). \textit{Computational Physics}. CreateSpace Independent Publishing Platform.

\bibitem{ghez2008-sgrA}
Ghez, A. M., et al. (2008). Measuring Distance and Properties of the Milky Way's Central Supermassive Black Hole with Stellar Orbits. \textit{The Astrophysical Journal}, 689(2), 1044-1062.

\bibitem{schneider2006-lensing}
Schneider, P., Kochanek, C. S., \& Wambsganss, J. (2006). \textit{Gravitational Lensing: Strong, Weak and Micro}. Springer-Verlag Berlin Heidelberg.

\bibitem{lisa-consortium2017}
Amaro-Seoane, P., et al. (LISA Consortium). (2017). Laser Interferometer Space Antenna. \textit{arXiv preprint arXiv:1702.00786}.

\bibitem{ligo2016-detection}
Abbott, B. P., et al. (LIGO Scientific Collaboration and Virgo Collaboration). (2016). Observation of Gravitational Waves from a Binary Black Hole Merger. \textit{Physical Review Letters}, 116(6), 061102.

\bibitem{maxwell1873-treatise}
Maxwell, J. C. (1873). \textit{A Treatise on Electricity and Magnetism}. Oxford: Clarendon Press.

\bibitem{debroglie1924-thesis}
de Broglie, L. (1924). Recherches sur la théorie des quanta (Researches on the quantum theory). \textit{Annales de Physique}, 10(2), 22-128.

\bibitem{hubble1929}
Hubble, E. (1929). A relation between distance and radial velocity among extra-galactic nebulae. \textit{Proceedings of the National Academy of Sciences}, 15(3), 168-173.

\bibitem{goldhaber2001-dilation}
Goldhaber, G., et al. (The Supernova Cosmology Project). (2001). Timescale Stretch Parameterization of Type Ia Supernova B-band Light Curves. \textit{The Astrophysical Journal}, 558(1), 359-368.

\bibitem{penzias1965-cmb}
Penzias, A. A., \& Wilson, R. W. (1965). A Measurement of Excess Antenna Temperature at 4080 Mc/s. \textit{The Astrophysical Journal}, 142, 419-421.

\bibitem{alpher1948-bigbang}
Alpher, R. A., Bethe, H., \& Gamow, G. (1948). The Origin of Chemical Elements. \textit{Physical Review}, 73(7), 803-804.

\bibitem{steinhardt2007-cyclic}
Steinhardt, P. J., \& Turok, N. (2007). \textit{Endless Universe: Beyond the Big Bang}. New York: Doubleday.

\bibitem{milgrom1983-mond}
Milgrom, M. (1983). A modification of the Newtonian dynamics as a possible alternative to the hidden mass hypothesis. \textit{The Astrophysical Journal}, 270, 365-370.

\bibitem{bertone2005-review}
Bertone, G., Hooper, D., \& Silk, J. (2005). Particle dark matter: Evidence, candidates and constraints. \textit{Physics Reports}, 405(5-6), 279-390.

\bibitem{bohm1952-interpretation}
Bohm, D. (1952). A Suggested Interpretation of the Quantum Theory in Terms of "Hidden Variables". I. \textit{Physical Review}, 85(2), 166-179.

\bibitem{maggiore2007-gw}
Maggiore, M. (2007). \textit{Gravitational Waves: Volume 1: Theory and Experiments}. Oxford: Oxford University Press.

\bibitem{tonomura1989-electron}
Tonomura, A., Endo, J., Matsuda, T., Kawasaki, T., \& Ezawa, H. (1989). Demonstration of single-electron buildup of an interference pattern. \textit{American Journal of Physics}, 57(2), 117-120.

\bibitem{larson2023-jwst-bh}
Larson, R. L., et al. (2023). A CEERS Discovery of an Overmassive AGN in a Low-mass Galaxy at z = 5.5. \textit{The Astrophysical Journal Letters}, 953(2), L29.

\bibitem{labbe2023-jwst-galaxies}
Labbé, I., et al. (2023). A population of red candidate massive galaxies ~600 Myr after the Big Bang. \textit{Nature}, 616(7956), 266-269.

\bibitem{subrayan2023-barbie}
Subrayan, B., et al. (2023). AT 2021lwx: the most luminous transient ever observed?. \textit{Monthly Notices of the Royal Astronomical Society}, 526(2), 1623-1634.

\bibitem{maiolino2024-jades-bh}
Maiolino, R., Scholtz, J., et al. (2024). A small and vigorous black hole in the early Universe. \textit{Nature}, 625, 477-480.

\bibitem{carniani2024-z14}
Carniani, S., Hainline, K., et al. (2024). Spectroscopic confirmation of two luminous galaxies at a redshift of ~14. \textit{Nature Astronomy}, 8, 1063-1070.

\bibitem{lu2024-jwst-lcdm-tension}
Lu, S., Frenk, C. S., et al. (2024). A comparison of pre-existing LambdaCDM predictions with the abundance of JWST galaxies at high redshift. \textit{arXiv preprint arXiv:2406.02672}.

\bibitem{arp1987-quasars}
Arp, H. (1987). \textit{Quasars, Redshifts and Controversies}. Berkeley: Interstellar Media.

\end{thebibliography}

\end{document}