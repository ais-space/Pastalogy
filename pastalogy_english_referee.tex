\documentclass[pdflatex,sn-mathphys-num,referee]{sn-jnl}
\usepackage[utf8]{inputenc}
\usepackage[T1]{fontenc}
\usepackage{amsmath}





\title[Pastalogy]{Pastalogy: An Axiomatic Theory of the Universe}

\author*[1]{\fnm{Vladimir} \sur{Sitnikov}}\email{montenegrofsm@google.com}

\affil*[1]{\orgdiv{Independent Researcher}, \city{Bar}, \country{Montenegro}}

\abstract{This paper presents Pastalogy, a new axiomatic theory of the universe developed from a single assumption: matter is divisible, but not infinitely so. From this simple axiomatic principle, all other propositions of the theory are deductively derived, including the nature of fundamental particles---pastons, the concepts of space and time, and the explanation of emergent phenomena. Within the framework of Pastalogy, gravity is interpreted as a pressure effect from ubiquitous background pastons, light as ordered clusters of pastons, and wave-particle duality is explained by the interaction of particles with an interfering background wave front. Cosmological puzzles such as dark matter and dark energy are resolved without introducing additional entities: dark matter is explained by the presence of large pastons and their aggregates, while dark energy is rejected as unnecessary as an accelerated expansion of the universe is not posited. The theory offers several testable predictions that differ from those of standard physics, including the non-identity of gravitational and inertial mass for superdense objects, the specific nature of "gravitational wave" signals from merging superdense systems, and the persistence of interference in slit experiments even when the particle's path is known with certainty. Pastalogy is positioned not as a branch of physics but as a standalone axiomatic science aiming to construct a complete and consistent picture of the world from its fundamental logical principles.}

\keywords{axiomatic theory, fundamental physics, gravity, wave-particle duality, cosmology}

\begin{document}

\maketitle

\section{Introduction}\label{sec:introduction}

\subsection{Motivation and Methodology}\label{subsec:motivation-methodology}

Despite its undeniable successes, contemporary physics faces a number of fundamental problems and ambiguities that cast doubt on the completeness and consistency of current paradigms. The existence of ineliminable "fields" as primary entities, the difficulties in synthesizing general relativity \cite{einstein1916} and quantum mechanics \cite{bohr1928}, and the need to introduce hypothetical "unexplained entities" such as dark matter \cite{rubin1980} and dark energy \cite{riess1998}, all point to potential gaps in our understanding of the universe. These challenges motivate the search for more fundamental, logically coherent, and axiomatic explanations of the universe's structure.

This paper introduces Pastalogy, a new axiomatic science that offers a radically different approach to comprehending reality. In contrast to empirical physics, which proceeds "bottom-up"---from observations and experiments to the formulation of laws and models---Pastalogy follows a deductive "top-down" path, starting from a single assumption and logically deriving the entire picture of the universe from it. This approach is analogous to axiomatic systems in mathematics, such as Euclidean geometry \cite{euclid-elements-heath}, where a complete system of knowledge is constructed from a few postulates.

We view the universe as a "black box" of sorts \cite{wiener1948}---a vast and complex object with unknown implementation mechanisms. Physics aims to understand its behavior by analyzing observed responses, which corresponds to the classical empirical approach in science \cite{popper1959}. Pastalogy, in contrast, attempts to "look inside" this box by positing the most basic principles of its construction and then deductively deriving how these principles give rise to observed phenomena. In the long run, we anticipate a "convergence" between Pastalogy and physics, where the empirical data obtained by physics will align with the logical deductions of Pastalogy, forming a unified, complete, and consistent picture of the world.

\subsection{Objective of the Paper}\label{subsec:purpose}

The primary objective of this paper is to present Pastalogy as a new axiomatic theory derived entirely from a single assumption. We aim to demonstrate the logical rigor and consistency of the deductive inferences that allow for the explanation of a wide range of physical phenomena and cosmological puzzles, often proposing interpretations that diverge from standard models.

\subsection{Introduction to Terminology}\label{subsec:terminology}

To adequately present the axiomatic basis of Pastalogy, new and specific terms are introduced, directly borrowed from the Pastafarian worldview. The fundamental, indivisible particles that constitute all matter are named "pastons" (in contrast to the "atoms" of Democritus \cite{kirk1983-democritus} or the "monads" of Leibniz \cite{leibniz1989-monadology}, as Pastalogy provides its own rationale for the nature of these particles). A separate category of pastons, "teftons," is also defined. A tefton is a paston of sufficient size to be influenced by the pressure of smaller background pastons, a process that can cause teftons to aggregate. A special role is played by "teftels"--- \textbf{pupermassive} black holes that, within the context of Pastalogy, possess the unique ability to generate new local "universes," similar to our own observable "Universe," through mutual collision and destruction. Other terms, such as "gravitational shadow" and "gravitational viscosity," will be explained later in the paper.

The choice of this terminology and the name of the science itself---"Pastalogy"---is not accidental and holds significant methodological importance. Unlike theology, which claims scientific status but lacks a scientific method and relies on faith and "spiritual experience," Pastalogy offers a strictly axiomatic, logically derivable, and scientifically verifiable approach to understanding the universe. Thus, through irony and allusion to the worldview articulated in \textit{The Gospel of the Flying Spaghetti Monster} \cite{henderson2006}, we assert that the universe itself, with all its laws and phenomena, is a manifestation of the Flying Spaghetti Monster, represented through a system of deterministic interactions among pastons. This "materialistic-religious" concept does not diminish the scientific rigor of Pastalogy but rather underscores its ambition to provide a fundamental explanation of reality derived from the simplest beginnings, in accordance with the principle of Occam's razor \cite{occam-razor-sep}.

\subsection{Structure of the Paper}\label{subsec:structure}

This paper is organized as follows: Section~\ref{sec:fundamentals} describes the fundamental principles of Pastalogy, including its single axiomatic assumption and the resulting properties of pastons. Section~\ref{sec:space-time} is devoted to the interpretation of space and time within the theory. Section~\ref{sec:emergent-cosmology} outlines emergent phenomena and cosmological consequences, such as the nature of gravity, light, the resolution of key cosmological problems, and wave-particle duality. Section~\ref{sec:predictions} presents a list of testable consequences and predictions of Pastalogy, some of which directly contradict the propositions of standard physics. The concluding Section~\ref{sec:discussion-conclusion} contains a discussion of the theory's strengths, its place within the scientific paradigm, and directions for future research.

\section{Fundamental Principles of Pastalogy}\label{sec:fundamentals}

\subsection{The Single Assumption (Axiom)}\label{subsec:axiom}

At the foundation of Pastalogy lies a single fundamental assumption, which serves as the axiom from which the entire subsequent picture of the universe is deductively constructed. This axiomatic principle is formulated as follows: matter is divisible, but not infinitely so. This assumption is rooted in Pastalogy's fundamental conviction in the comprehensibility of the world and in causality as the basis of logic and reality, which distinguishes the theory from approaches that deny the purpose of scientific inquiry.

This assumption acknowledges the obvious, observable property of matter---its divisibility. We can successively divide objects into ever-smaller components: matter into molecules, molecules into atoms, atoms into elementary particles. However, unlike concepts that presuppose infinite division, Pastalogy postulates the existence of a finite limit to this divisibility. This implies that there exists some indivisible and structureless element, upon reaching which further division is impossible. It is this elementary, indivisible object that constitutes the foundation of all matter in the universe.

\subsection{The Nature of Fundamental Particles---Pastons}\label{subsec:pastons}

From the axiom of the finite divisibility of matter, we arrive at the definition of the fundamental particle: the paston.

A paston is a truly elementary, structureless object. It is not composed of any smaller components and cannot be divided. This means that a paston has no internal dynamics resulting from the motion of its constituent parts; all its properties are intrinsic and inherent. Unlike other concepts that posit fundamental particles without internal structure (e.g., leptons or quarks in the Standard Model \cite{pdg2024}), Pastalogy does not attribute complex properties such as electric charge, spin, or color to pastons as their fundamental attributes. Such complex properties can only arise from structural complexity and, in Pastalogy, are interpreted as emergent characteristics arising from specific configurations and interactions of multiple pastons in more complex aggregates.

From the fact that pastons can interact with each other, it logically follows that they possess size and shape. Interaction occurs through collisions, as will be demonstrated later, which is impossible for point-like objects. It is posited that pastons may have various shapes and sizes, which could in turn explain the diversity of their aggregates.

The key property of a paston is its inertial mass. Within Pastalogy, inertial mass is defined as an intrinsic, inherent property of the paston itself, directly proportional to its volume (size). For any composite object---be it an elementary particle, an atom, or a planet---its inertial mass is simply the sum of the inertial masses of all its constituent pastons.

Since pastons are structureless and indivisible, it logically follows that they are immutable and eternal. A paston cannot be created or destroyed; it exists outside of time and is not subject to decay. This, in turn, leads to the conclusion that the universe, being composed of these eternal and immutable pastons, is itself eternal and infinite.

Between collisions, pastons move in a straight line at a constant velocity. A change in their motion can only occur as a result of an interaction with other pastons. The concept of momentum, as the product of a paston's inertial mass and its velocity, is fundamental to describing their motion and interactions.

\subsection{The Single Fundamental Interaction}\label{subsec:interaction}

Since pastons are structureless, indivisible, and cannot possess internal forces or fields, the only physically possible way for them to interact is through a perfectly elastic collision. This means that when two pastons collide, there is a perfect conservation of momentum without any loss of energy into internal states (as a structureless paston can have no internal states). Any rotation of a paston that may arise from a collision also does not lead to a loss of energy, as the rotational energy is fully converted back into kinetic energy in subsequent collisions.

All other observed forces and interactions in nature---electric, magnetic, nuclear---are not fundamental. In Pastalogy, they are interpreted as emergent, statistical effects arising from the vast number of successive, perfectly elastic collisions of pastons organized into more complex aggregates (such as electrons, protons, and other particles commonly referred to as fundamental).

\subsubsection{Momentum Transfer in Paston Collisions}\label{subsubsec:impulse-transfer}

Classical mechanics describes the perfectly elastic collision of composite bodies that possess internal structure. In this case, the process involves two phases: (1) compression of the bodies as they approach, during which kinetic energy is partially converted into internal deformation energy; and (2) restoration of shape, in which the internal energy is converted back into kinetic energy, producing the characteristic \emph{rebound} formulated as the law of equal angles of incidence and reflection. Thus, the classical model of collision relies on the existence of an internal "spring" within a material body \cite{feynman-lectures-v1}.

For pastons, as indivisible and structureless elements of matter, this model is inapplicable. A paston has no internal degrees of freedom that could accumulate and return the energy of an impact. Consequently, a collision between pastons is an \emph{instantaneous transfer of momentum} without a phase of deformation and restoration. As a result:

\begin{enumerate}
  \item The contact time between pastons during a collision is zero.
  \item There is no reflection effect: momentum is always transferred along the direction of motion of the striking paston.
  \item For pastons of equal mass, the moving paston comes to a complete stop at the moment of impact, while the second paston continues to move with the same velocity along a parallel trajectory from its own coordinates.
  \item For pastons of different masses, the distribution of momentum obeys the law of conservation, but without the possibility of a "spring-like return": the more massive paston acquires a lower velocity, inversely proportional to its mass, and the less massive one cannot exceed the velocity of the striking paston.
\end{enumerate}

Thus, in Pastalogy, a \textbf{perfectly elastic collision} is understood not as an elastic rebound but as a direct and instantaneous transfer of momentum between structureless elements. This distinction from classical mechanics is fundamental and must be taken into account in all subsequent constructions. It should be noted that, since pastons possess size and shape, off-center collisions will inevitably lead to the emergence of rotation. This rotational degree of freedom complicates the dynamics of interactions, as kinetic energy can be redistributed between translational and rotational motion. However, this does not violate the principle of a perfectly elastic collision, as the total energy of the system is conserved. Moreover, it is this fundamental possibility of rotation that may underlie the observable property of particles known as spin, as well as determine the specifics of their interaction with asymmetric fluxes of background pastons (manifested, for example, in the peculiarities of their trajectories near large aggregates of matter). A detailed analysis of the dynamics of rotating pastons is beyond the scope of this overview paper but represents a key direction for future research.

\subsection{Determinism and Incomputability}\label{subsec:determinism}

At the micro-level, Pastalogy is a strictly deterministic theory. The motion of each individual paston and the outcome of each collision are completely predetermined by the initial conditions and the laws of elastic impact. However, due to the infinite number of pastons in the universe and the unimaginable number of their sequential interactions, the behavior of the system at the macro-level becomes incomputable. This incomputability generates the apparent randomness we observe in quantum phenomena, but this is not true indeterminacy, merely a consequence of the complexity and the impossibility of tracking all individual interactions.

\subsection{Energy}\label{subsec:energy}

In Pastalogy, energy is considered a mathematical abstraction rather than an independent physical entity or substance. It represents a way of describing the characteristics of motion and the processes occurring with pastons. Kinetic energy, in particular, is a quantitative measure of the motion of pastons and their ability to perform work through collisions.

\section{Space and Time in Pastalogy}\label{sec:space-time}

\subsection{Space}\label{subsec:space}

In traditional physics, space is often regarded as an independent entity possessing its own properties and capable of curving or expanding \cite{mtw1973}. Pastalogy, however, proposes a radically different understanding. Space is defined as "Nothing"---it is not a physical object, a field, a "fabric," or a substance. It merely represents the locus for the existence of matter, i.e., pastons.

In its essence, space possesses no intrinsic physical properties. It has no mass, energy, density, curvature, or any internal structure. A consequence of this definition is the impossibility for space itself to "curve" or "expand." Observed effects that are interpreted in other theories as deformations of spacetime (such as gravitational phenomena or cosmological redshift) are explained in Pastalogy through changes in the density, distribution, or dynamics of the pastons themselves.

Between pastons lies an absolute void. This means that space is filled with nothing but pastons. Any interaction occurs exclusively through direct, perfectly elastic collisions between pastons. Thus, Pastalogy proposes a universe where there is only discrete matter (pastons) and empty space between them, and all observable phenomena are the result of their mechanical motion and collisions.

\subsection{Time}\label{subsec:time}

Similar to space, time in Pastalogy is a mathematical abstraction, not an independent physical entity or dimension. Time does not "flow," "exist," or possess its own properties or direction.

Instead, time is defined as a measure of the sequence of changes and the characteristics of processes involving pastons. We perceive time as "flowing" because we observe a succession of movements and interactions among pastons. "Past," "present," and "future" are exclusively mental constructs arising from our perception of a sequence of events. In reality, there is only a continuously changing configuration of the universe, consisting of pastons in constant motion and interaction.

The absence of physical time implies that there are no "locations" in time to which one could travel, nor is there a "current" that could slow down or speed up on its own or for any reason. All effects that other theories associate with time dilation (e.g., relativistic time dilation) must be explained in Pastalogy as changes in the frequency or speed of physical processes involving pastons, rather than as a change in the properties of time itself.

\section{Emergent Phenomena and Cosmology in Pastalogy}\label{sec:emergent-cosmology}

\subsection{Gravity as a Pressure Effect}\label{subsec:gravity-pressure}

In Pastalogy, gravity is not considered a fundamental force of interaction or a curvature of the spacetime continuum. Instead, it is interpreted as an emergent pressure effect arising from multiple perfectly elastic collisions.

At first glance, this mechanism may appear similar to historical "push gravity" theories, the most famous of which is the theory of Georges-Louis Le Sage \cite{lesage-edwards2014}. However, the distinction between Pastalogy and Le Sage's model is both fundamental and methodological. Le Sage's theory was an \textit{ad hoc} hypothesis proposed specifically to explain gravity, requiring the postulation of special "ultramundane corpuscles" with specific properties (such as perfectly inelastic collisions, which led to the intractable problem of immense heating of bodies). In Pastalogy, by contrast, gravity as a pressure effect is not an initial postulate but rather an \textbf{inevitable deductive consequence} of the single axiom of the finite divisibility of matter. The properties of pastons and their perfectly elastic collisions are not assumptions made to explain gravity but are logical deductions from this fundamental principle. Thus, in Pastalogy, gravity is not the goal of the theory but an emergent phenomenon that arises organically from its foundations.

The universe is permeated by a countless multitude of background pastons, which are in constant motion, traveling in all possible directions. These background pastons collide with any objects composed of other pastons.

The effect traditionally and erroneously termed "attraction" is explained in Pastalogy as a pressure imbalance created by the flux of background pastons. When two or more objects composed of pastons are in proximity, they begin to mutually "shield" each other from a portion of the incoming flux of background pastons. Each object casts a "gravitational shadow"---a region into which background pastons penetrate in smaller numbers or with lower intensity due to collisions with the pastons of the shielding object.

Thus, every object is subjected to the impact of background pastons from all sides. However, from the direction of another nearby object, this impact is slightly reduced due to mutual shielding. The resulting imbalance of impacts leads to the convergence of the objects, creating the observable effect of gravity.

The concept of a body's "gravitational shadow" is key. It is the total projected area of all pastons constituting the body onto a plane perpendicular to the flux of background pastons. The magnitude of this shadow, which is proportional to the object's inertial mass, determines how effectively the body shields the background pastons. However, at close distances and when analyzing the impact on objects located within other bodies (e.g., on the surface or deep inside a planet), the spatial distribution of the gravitational shadows from the constituent pastons must be considered. This distribution affects the resultant gravitational impact.

This concept of gravity as a pressure generated by perfectly elastic collisions is consistent with the principles of mechanics and avoids the need to introduce unsubstantiated "forces" or "fields," which aligns with the minimalist approach of Pastalogy.

\subsection{Non-Identity of Gravitational and Inertial Mass}\label{subsec:mass-nonequivalence}

In standard physics, inertial and gravitational mass are considered identical---this is the principle of equivalence \cite{einstein1920-relativity}. In Pastalogy, however, under certain conditions, inertial and gravitational mass are not strictly identical, a direct consequence of the mechanism of gravity as a pressure from background pastons (see Section~\ref{subsec:gravity-pressure}).

The inertial mass of an object, as previously defined (see Section~\ref{subsec:pastons}), is the sum of the inertial masses of all its constituent pastons. It is a measure of the "quantity of matter" and its resistance to a change in motion.

The term "gravitational mass" is incorrect in Pastalogy and is not used, as mass itself plays no role in the gravitational effect. The effect of gravity is determined exclusively by the geometric properties of pastons---their ability to create a "gravitational shadow" or an effective cross-section for collisions with background pastons. The correlation between this geometric property and inertial mass is observed only in diffuse bodies and is a regularity, not a causal relationship. Thus, in Pastalogy, when referring to mass, it is always exclusively the inertial mass that is meant, as no other kind exists.

To explain the effect traditionally associated with gravitational mass, the concept of shielding capability is used. This capability is determined by the extent to which an object intercepts the flux of background pastons, thereby creating "gravitational pressure" on other objects.

For ordinary, not overly dense objects, the shielding capability created by each paston is cumulative, and the aggregate effect is indeed approximately proportional to the total number of pastons, i.e., the inertial mass. However, the situation changes drastically when considering superdense objects, such as neutron stars or black holes \cite{shapiro-teukolsky1983}.

Inside such objects, pastons are packed extremely densely. This leads to the phenomenon of mutual shielding of pastons. Background pastons penetrating such a superdense object may collide with some pastons before they can reach others located deeper inside. As a result, some internal pastons are shielded from the external flux of background pastons not only by other external objects but also by their own "neighbors" within the same superdense aggregate. The gravitational shadows of the pastons that make up superdense objects overlap one another, leading to a reduction in the total gravitational shadow.

The consequence of this mutual shielding is that the total shielding effectiveness of a superdense body is less than the sum of the shielding effectiveness of all its constituent pastons, were they in a sparse state. Thus, although the inertial mass of the object (the total number of pastons) remains unchanged, its ability to create gravitational pressure (the "gravitational effect") decreases relative to the expected proportionality.

The logical conclusion from this mechanism is that the shielding effectiveness of superdense objects is not proportional to their inertial mass and will be less than expected. This effect becomes more pronounced as the density and size of the object increase. This prediction is one of the key distinctions of Pastalogy from the equivalence principle of general relativity and offers a testable method for falsifying the theory (see Section~\ref{subsec:mass-inequivalence}).

\subsection{Mathematical Model of Gravity}\label{subsec:gravity-model}

In Pastalogy, gravity is interpreted as an emergent pressure effect from an isotropic flux of background pastons, which creates a pressure asymmetry due to shielding by bodies. The key quantity is the effective gravitational shadow of a body, \( A \) (in m\(^2\)), which is its projected area against the flux front, determining the extent to which the body intercepts the background flux.

For two bodies with shadows \( A_1 \) and \( A_2 \), where the second body (the satellite) has an inertial mass of \( M_2 \) and the distance between them is \( r \) (with \( r \) being much larger than the dimensions of the bodies), the acceleration of the satellite is described by the formula:
\[
a_2(r) = \frac{4\pi p_0 A_1 A_2}{M_2 r^2}
\]
where \( p_0 \) is the background pressure of the paston flux (in N/m\(^2\)). This formula reflects that the gravitational effect is determined by the geometric properties of the shadows rather than by mass as such.

For diffuse bodies, where \( A \propto M \), the model reproduces Newton's law of gravitation \cite{newton1687-principia, halliday2013-fundamentals} with an effective constant:
\[
G_{\text{eff}} = 4\pi p_0 k_\ell^2
\]
where \( k_\ell \) is the proportionality coefficient between \( A \) and mass \( M \). By adjusting \( p_0 \) and \( k_\ell \) such that \( G_{\text{eff}} = G_N \), the model is brought into agreement with observations for ordinary objects.

For superdense objects (e.g., black holes), the dependence of \( A(M) \) becomes nonlinear, for instance,
\[
A \propto M^\alpha, \quad \alpha < 1
\]
with a first-order approximation of \( \alpha \approx 2/3 \). This leads to the "gravitational mass" of such objects always being less than their inertial mass, which directly contradicts the strong equivalence principle in general relativity and provides a testable prediction (see Section~\ref{subsec:mass-inequivalence}).

Paths for the future development of the model include:
\begin{itemize}
    \item refining the \( A(M) \) dependence through numerical simulations \cite{newman2013-computational} (e.g., Monte Carlo methods for paston collisions or 3D N-body simulations of aggregates),
    \item determining the parameters \( p_0 \) and \( \alpha \) from observations (by comparing the orbital accelerations of diffuse and compact tracers at the same distances from a center of mass),
    \item developing Bayesian methods for fitting parameters to astrophysical observation data (e.g., the orbits of stars and pulsars around Sgr A* \cite{ghez2008-sgrA}, gravitational lensing \cite{schneider2006-lensing}, or future data from LISA \cite{lisa-consortium2017}).
\end{itemize}
These steps will allow for the refinement of quantitative predictions and enhance the empirical testability of the theory.

\subsection{Gravitational Waves}\label{subsec:gravitational-waves}

In classical physics, "gravitational waves" are interpreted as perturbations of the spacetime metric that propagate at the speed of light \cite{ligo2016-detection}. In Pastalogy, where space and time are not physical entities, such an interpretation is inadmissible. Instead, the observed signals, such as those detected by the LIGO/Virgo interferometers, are explained as periodic variations in the collective gravitational shadow generated by the dynamics of superdense objects.

As shown in Section~\ref{subsec:gravity-pressure}, gravity is the result of pressure from background pastons, and the effectiveness of this pressure depends on the geometric shielding by the pastons that constitute an object. In the case of systems with two or more superdense bodies, such as neutron stars or black holes, in tight orbits and actively inspiraling (e.g., during a merger), their mutual configuration changes continuously.

When two such superdense objects orbit each other, their collective shielding capability, as projected in the direction of an observer situated in or near the orbital plane, varies periodically. At certain moments, when their gravitational shadows maximally overlap from the observer's line of sight, the system's shielding capability is at a minimum. Conversely, when the overlap is minimal or absent, the collective shadow is at its maximum. The intensity of the signal thus depends on the viewing angle, reaching its maximum in the orbital plane and diminishing with deviation from it.

These periodic changes in shielding capability generate variations in the background paston pressure, which propagate from the orbiting system to a distant observer. It is precisely these variations in the local background paston pressure that are registered by gravitational interferometers and interpreted as "gravitational waves." They cause microscopic but periodic deformations of the detectors (e.g., interferometers), which correspond to the observed signals.

It is important to emphasize that this is a statistical effect, driven by changes in the geometry of the distribution and overlap of pastons in moving superdense systems. It is not "waves in the fabric of spacetime" but rather perturbations in the local density or intensity of the background paston flux, caused by the dynamics of the shielding system (the superdense objects). This explanation is consistent with the principle of no force-based interaction at the paston level (see Section~\ref{subsec:gravity-pressure}) and with the emergent nature of all complex interactions.

\subsection{The Nature of Light and "Photon Aging"}\label{subsec:light-aging}

In Pastalogy, light is neither a dual entity (simultaneously a wave and a particle) \cite{debroglie1924-thesis} nor is it interpreted as an oscillation of an electromagnetic field in a vacuum \cite{maxwell1873-treatise}. Its nature is explained through the mechanics of pastons.

\paragraph{The Photon as a Cluster of Pastons.}
A photon is an ordered cluster of pastons moving coherently in a single direction. Upon emission by an atom (more precisely, by an electron within it), a portion of the background pastons is organized into a stable configuration and acquires energy, which is then transported as the photon. The mechanism of this organization is as yet unknown, but the result is of fundamental importance: a photon is not a "special" particle but an ordered collective of ordinary pastons.

\paragraph{The "Swap" Mechanism.}
While traveling through the cosmos, a photon collides with chaotically moving background pastons. As a result of a perfectly elastic collision, an exchange of momentum occurs:
\begin{enumerate}
  \item A paston from the photon's composition transfers its momentum to a background paston and leaves the cluster.
  \item The new paston accepts the momentum and becomes part of the photon's coherent motion, but begins its path from its own initial position in space.
  \item Each such event slightly alters the positions of the pastons within the cluster: some move closer to the center, others remain at the same distance, and some move farther out. On average, this leads to a gradual outward displacement of some pastons, causing the linear dimensions of the cluster to blur, despite the parallel nature of their trajectories being maintained.
\end{enumerate}

\paragraph{Redshift as a Consequence.}
Since in Pastalogy the wavelength of light is related to the linear size of the photon, the growth in the cluster's dimensions from multiple collisions implies a lengthening of the wave. An observer perceives this as \emph{redshift} \cite{hubble1929}. This mechanism, which explains redshift as a statistical process of \emph{photon aging}, does not require the expansion of space or the hypothesis of dark energy. It should be noted that this concept belongs to a class of models historically known as "tired light" hypotheses. However, unlike early \textit{ad hoc} models, which failed to pass key observational tests---such as the absence of the predicted time dilation effect in the light curves of Type Ia supernovae \cite{goldhaber2001-dilation}---"photon aging" in Pastalogy is a direct deductive consequence of the mechanics of paston collisions. This opens up the possibility of constructing a more sophisticated model that could potentially be consistent with the observational data where simpler hypotheses have failed.

\paragraph{Terminal Aging.}
In the limiting case, a photon does not disintegrate into an incoherent flux but gradually diffuses into a cloud of pastons. The key factor in this process is not so much the increase in its linear dimensions as the decrease in the local density of pastons within the cluster. When the density becomes too low, the cloud loses its ability to transfer the discrete momentum required for interaction with matter and effectively ceases to possess the properties of a photon. On cosmological timescales, this form of terminal aging may be associated with the phenomenon of the \emph{cosmic microwave background radiation}, which in Pastalogy is interpreted as a statistical background of scattered pastons that have lost their ability to exist in the form of photons. On cosmological scales, the observed cosmic microwave background is the statistical result of the terminal aging of a multitude of photons.

\subsubsection*{Cosmic Microwave Background}\label{subsubsec:relic-radiation}
In standard cosmology, the cosmic microwave background is interpreted as the "echo" of a singular Big Bang \cite{alpher1948-bigbang, penzias1965-cmb} and is considered evidence of a hot beginning for the universe.
In Pastalogy, it receives a different explanation: the cosmic microwave background is not a memory of a singular event but a thermal background that arises from the repeated scattering and gradual aging of light through its interaction with pastons.
Such a process is universal and occurs everywhere, ensuring the uniformity of the observed background without the need to postulate a unique act of creation.
Thus, in Pastalogy, the cosmic microwave background reflects not a beginning, but the continuous dynamics of the interaction of light with the fundamental medium of pastons.

\subsection{Resolution of Cosmological Puzzles}\label{subsec:cosmological-solutions}

Through its fundamental principles and mechanisms, Pastalogy offers unique and consistent explanations for a number of cosmological puzzles that pose significant challenges to the standard model.

\subsubsection{The Big Bang as a Mundane Event}\label{subsubsec:big-bang}

It should be noted that the cosmic microwave background, typically interpreted as the residual heat of a singular Big Bang, is understood in Pastalogy as a natural result of continuous scattering processes and therefore cannot be considered proof of a singular origin for the universe.
In standard cosmology, the Big Bang concept describes the beginning of the universe from a singularity, positing a single event from which the expansion of space and all matter originated. Pastalogy, by rejecting relativistic theories and their postulates, points to the absence of any need for a singularity to explain the origin of large-scale structures. Instead of expending time and effort on calculating and adding "crutches" to an outdated theory, Pastalogy offers a different vision: the Big Bang is interpreted as a local, yet colossal and mundane, event initiated by the Invisible Flying Spaghetti Monster as part of its cyclical creation of the cosmos. Although the idea of a cyclical universe has been considered in other cosmological models \cite{steinhardt2007-cyclic}, Pastalogy proposes a unique mechanistic scenario based on the collision of giant aggregates of matter, a process continually occurring in various corners of an infinite universe.
This process is part of a continuous cycle of matter composed of eternal and indestructible pastons. The essence of the Big Bang mechanism in Pastalogy is as follows:
\begin{enumerate}
    \item \textbf{Gravitational Condensation and Formation of Teftels:} In an eternal and infinite universe, matter (pastons) gradually falls into gravitational traps. This leads to its condensation into various objects, from small to large, which are eventually absorbed by black holes. These black holes, upon reaching gigantic sizes and maturing to participate in the act of creation, are distinguished as a special category and are named "teftels" in Pastalogy (from the Russian \textit{teftel'}, pl. \textit{tefteli}, transl. "meatball," a term also used in Pastafarian scripture \cite{henderson2006}). As time is unlimited for them, teftels continue to "fatten up," consuming everything within their gravitational reach.
    \item \textbf{Motion Towards Collision:} As a result of this process, giant voids may form, containing only a few, but exceptionally massive, teftels moving in uniform rectilinear motion. Over vast cosmic distances, these giant objects, overcoming their colossal inertia, begin to move very slowly toward each other. Gradually, their closing speed and the precision of their aim increase, turning them into self-guiding "projectiles."
    \item \textbf{The Local Big Bang as an Act of Creation:} Sooner or later, a head-on (or near head-on) collision between two such giant teftels occurs. This event is the culmination of the process initiated by the FSM and represents the creation of a new region of the universe. Despite its locality on the scale of an infinite universe, it is a colossal and energetically saturated event. The impact generates a powerful dispersal of matter in all directions. This dispersal is non-uniform due to the mass difference between the colliding teftels and the imprecision of their mutual targeting.
    \item \textbf{Formation of the Observable Picture of the Universe:}
    \begin{itemize}
        \item \textbf{Dispersal of Small Particles:} The smallest particles---background pastons---are dispersed in the greatest numbers. It is such collisions throughout the universe that constantly replenish the background of space with these ubiquitous pastons.
        \item \textbf{Formation of New Objects:} The largest, densest clumps of matter are dispersed the least and immediately begin to form into aggregates. Among them, new black holes of all sizes are formed, though many times smaller than the colliding giants. From these scattered pastons and their aggregates, stars, galaxies, and the large-scale structures we observe are eventually formed. The picture of a local region of the universe after such a collision would look exactly as we observe it in the real world, with its inherent non-uniformities and structures. These non-uniformities in the large-scale picture of the observable part of the universe could potentially be used to reconstruct the teftel collisions that lie at their origin.
    \end{itemize}
\end{enumerate}
Thus, in Pastalogy, the Big Bang is not a unique beginning but a mundane, cyclical event of matter's rebirth, occurring within an eternal and infinite universe. It provides for the continuous formation and renewal of visible regions of the cosmos, explaining the observed motion of galaxies as a consequence of the initial momentum from such collisions, rather than from the expansion of space itself.

\subsubsection{The Nature of "Dark Matter"}\label{subsubsec:dark-matter}

In standard cosmology, "dark matter" is postulated to explain phenomena such as the anomalous rotation curves of galaxies \cite{rubin1980}---an observational puzzle that has also led to the development of alternative theories, for instance, Modified Newtonian Dynamics (MOND) \cite{milgrom1983-mond}. In Pastalogy, the need for either of these hypotheses is eliminated.
"Dark matter" in Pastalogy consists of vast, diffuse clusters of a special type of paston called teftons. A tefton is a paston that differs from background pastons in its size: a tefton is large enough to be subjected to gravitational pressure and to possess a shielding capability (a gravitational shadow) itself. A background paston, in contrast, is of insufficient size for this and serves merely as a momentum carrier, not being subject to significant gravitational influence. Teftons, like background pastons, move in uniform rectilinear motion between collisions. It is probable that stable "fundamental" particles known in standard physics are composed of teftons.
These teftons form enormous, diffuse halos around galaxies. Such a halo of teftons possesses a considerable total gravitational shadow and can be regarded as a gigantic cosmic object of very low density. Its order is limited only by the overall geometry of this halo.
The effect on bodies located within such a dark matter halo is termed gravitational viscosity. This effect is a direct gravitational influence arising from the distribution of gravitational shadows cast by the teftons comprising the halo. If a body is at the center of the halo, the gravitational influence on it will be uniform from all directions due to the symmetric distribution of gravitational shadows from the surrounding teftons. However, the closer the body is to the edge of the halo, the stronger the gravitational influence will become, as the imbalance of gravitational shadows will be more pronounced, and the net vector of this influence will be directed toward the center of the halo. It is this gradient of gravitational pressure that gives rise to the "viscosity."
Thus, a halo of teftons exerts a direct gravitational influence on ordinary matter while remaining invisible to all forms of electromagnetic radiation. This is explained by the fact that, being pastons, teftons have no internal structure and are incapable of the complex emergent interactions required for emitting or absorbing light. This explains why, according to current observational data, "dark matter" manifests itself only through its gravitational influence and has so far not been detected in direct detection experiments \cite{bertone2005-review}.

\subsubsection{The Absence of "Dark Energy"}\label{subsubsec:no-dark-energy}

The concept of "dark energy" was introduced to explain the apparent accelerated expansion of the universe \cite{riess1998}, which is inferred from the interpretation of additional redshift as an expansion of space. Since in Pastalogy the additional redshift is explained by the phenomenon of "photon aging" (see Section~\ref{subsec:light-aging})---which consists of an increase in the photon's wavelength due to the "swap" mechanism during multiple collisions with background pastons, leading to an increase in the photon's linear dimensions---the necessity for the existence of "dark energy" is entirely obviated.

In Pastalogy, the universe is static and infinite; it neither expands nor has a center. The observed redshift is a consequence of the increase in the linear dimensions of photons as they travel over vast distances, not of the recession of galaxies from one another or the accelerated expansion of space. Thus, Pastalogy offers a natural resolution to the problem of "dark energy," eliminating the need for a hypothetical entity that would cause the "expansion of Nothing."

\subsection{Wave-Particle Duality}\label{subsec:wave-particle-duality}

The phenomenon of wave-particle duality, which in quantum mechanics describes particles as simultaneously possessing the properties of both particles and waves, receives a different, more mechanistic explanation in Pastalogy. There is no need to postulate a dual nature of matter or a "collapse of the wave function," which remains one of the most enigmatic and contentious aspects of quantum mechanics.

It should be noted that this approach, which separates the particle from its guiding wave, bears a conceptual resemblance to "pilot-wave" theories, such as de Broglie--Bohm mechanics \cite{bohm1952-interpretation}. However, unlike these theories, which postulate the existence of an abstract, non-local "quantum potential" wave as a separate fundamental entity, Pastalogy offers a purely mechanistic and local picture. Within its framework, the "wave" is not an independent field but an emergent, physical perturbation in the medium of ubiquitous background pastons, generated by the motion of the particle itself.

In Pastalogy:
\begin{enumerate}
    \item \textbf{The "particle" as a cluster of pastons.} Fundamental "particles" (such as electrons, protons, and more complex formations) are stable clusters of pastons (see Section~\ref{subsec:light-aging}). These clusters possess a definite inertial mass (as the sum of the masses of their constituent pastons) and move through space, interacting with other pastons only via perfectly elastic collisions. Their discrete, localized nature corresponds to the "particle" aspect of the duality.
    \item \textbf{"Wave-like" behavior as an emergent effect.} The wave-like properties observed in such "particles" are not an intrinsic property of the particle itself but an emergent effect arising from its interaction with the background of ubiquitous pastons. As a cluster of pastons (a "particle") moves through space, it continuously collides with a multitude of background pastons moving in all directions. These collisions generate a regular wave front of background pastons that propagates around the moving particle. This wave front is not a "probability wave" but a real, physical perturbation in the medium of background pastons, propagating according to the laws of mechanics.
    \item \textbf{Wavelength and interaction with the medium.} The background wave front has a specific wavelength, which we identify with the wavelength of the particle that creates it. For this reason, when passing through a medium (e.g., a solid), the wave can be scattered if its wavelength is comparable to the distance between the particles of the medium. However, when passing through slits of sufficient size, the wave front remains a wave, as its scale is significantly larger than the size of the elements of the barrier.
    \item \textbf{Interference and diffraction.} When a "particle" passes through obstacles such as slits (in a double-slit experiment, see Section~\ref{subsec:slit-experiments}), this background wave front passes through all available slits and interferes, creating a characteristic interference pattern in the medium of background pastons behind the obstacle.
    \item \textbf{Detection.} The "particle" itself (the discrete cluster of pastons) continues its motion and eventually reaches a detector. However, its trajectory is not arbitrary; it is modulated by the maxima of the interfering wave front of background pastons. That is, the probability of detecting the particle at a particular point on the detector is determined by the intensity of the background paston wave front at that point. This explains why, although the particle is a discrete object, its arrivals at the detector are statistically distributed according to a wave-like pattern. Detection always registers a discrete cluster of pastons, not a "wave."
\end{enumerate}

Thus, in Pastalogy, wave-particle duality is not a mystical intrinsic property of particles but is logically derived from their nature as clusters of pastons and their mechanical interaction with the background medium. The "particle" always remains a particle, and the "wave" is a real, physical perturbation in the medium of background pastons that influences the particle's behavior.

\subsection{Other Particle Properties (Electric Charge, Spin, Color)}\label{subsec:particle-properties}

In addition to inertial mass and the wave-like properties manifested through interaction with background pastons, observed fundamental particles possess a range of other characteristics, such as electric charge, spin, and "color" (within the framework of quantum chromodynamics) \cite{pdg2024}. In Pastalogy, these properties are not postulated as being intrinsically inherent to pastons but are explained as high-level, emergent effects arising from specific configurations, complex internal dynamics, and the spatial arrangement of pastons in organized clusters.

\begin{itemize}
    \item \textbf{Electric Charge.} It is proposed that electric charge is the result of a specific stable, yet dynamic, asymmetry in the distribution and motion of pastons within a particle. This asymmetry creates specific patterns in the local pressure of background pastons or in their momentum exchange with other charged particles (see Section~\ref{subsec:gravity-pressure}). This may be related to peculiarities of the internal circulation of pastons or their geometric arrangement, which lead to a pressure that has a directional character.
    \item \textbf{Spin.} The spin of a particle, which is a fundamental characteristic, is described in Pastalogy as the result of the rotation of pastons within the cluster. This internal motion creates a specific gyroscopic effect and influences how the particle interacts with the background medium and other particles by transferring rotational momentum.
    \item \textbf{"Color" (Strong and Weak Interactions).} The strong and weak interactions, including the properties known as "color" in the Standard Model, share the same fundamental nature as gravity---they are manifestations of momentum exchange through paston collisions, creating pressure effects. However, unlike gravity, these interactions manifest with a different character and intensity at small scales due to the specific configurations and internal dynamics of the pastons forming the particles. The results of 3D computer modeling of these processes convincingly demonstrate these mechanisms, but the overview format of this paper is not suitable for describing these experiments.
\end{itemize}

It must be acknowledged that the detailed derivation and mathematical description of these complex properties from the fundamental assumption about pastons require significant further development of the theory. At this stage, Pastalogy offers a qualitative foundation for understanding these phenomena as consequences of complex mechanical interactions and configurations of the simplest elements, rather than as intrinsically inherent fundamental fields or irreducible quantum numbers.

\section{Predictions and Falsifiability of Pastalogy}\label{sec:predictions}

One of the most crucial features of any scientific theory is its ability to make testable predictions that can be experimentally or observationally confirmed or, conversely, refuted. Only the existence of such predictions allows for the distinction between a scientific theory and a metaphysical concept \cite{popper1959}. Despite its radically different fundamental nature, Pastalogy not only explains existing phenomena but also makes a series of unique predictions that differ from those of the standard physical model. These discrepancies offer avenues for the potential falsification of the theory.

\subsection{Non-Equivalence of Inertial and "Gravitational" Mass for Superdense Objects}\label{subsec:mass-inequivalence}

As described in detail (see Section~\ref{subsec:mass-nonequivalence}), in Pastalogy, the gravitational effect is not identical to inertial mass for superdense objects. This is a direct consequence of the "mutual shielding of pastons" mechanism within such bodies.

\paragraph{Prediction.} According to Pastalogy, for objects of extremely high density (e.g., neutron stars, black holes), their shielding effectiveness ("gravitational mass" in traditional terminology) will be less than their inertial mass. The higher the density and the larger the linear dimensions of the superdense object, the more pronounced this effect should be.

\paragraph{Methods of Falsification/Verification.} Testing this prediction requires high-precision astrophysical observations of systems that include superdense objects. If it were possible to conduct precise measurements of the gravitational influence of such an object (for example, through its effect on the orbits of other bodies in binary or multiple systems, or through the analysis of gas motion around it) and compare it with its inertial mass (which can be independently determined from the system's dynamic parameters, such as orbital period and velocities), then the discovery of an exact correspondence between them for all densities, up to the very highest, would constitute a direct refutation of Pastalogy's predictions.

Conversely, if even small deviations toward a reduced gravitational effect compared to inertial mass were detected for superdense objects, this would serve as strong confirmation of Pastalogy's predictions. Current observational data may be insufficient to detect such subtle discrepancies due to limitations in measurement accuracy or the need for long-term observations. However, the development of astrophysical instruments and techniques, such as more precise measurements of the orbital parameters of binary pulsars, accretion disks around black holes, or "gravitational-wave" signals from pre-merger systems, could potentially provide the necessary accuracy for such a test in the future.

\subsection{Predictions for Gravitational Perturbation Astronomy}\label{subsec:gravitational-astronomy}

The representation of gravity in Pastalogy as an emergent pressure effect of background pastons (see Section~\ref{subsec:gravity-pressure}) leads to specific, testable predictions that differ from the interpretation of registered phenomena in General Relativity. Whereas in standard physics the observed oscillations are considered "gravitational waves"---perturbations of the spacetime fabric (see Section~\ref{subsec:gravitational-waves})---in Pastalogy, these registered "gravitational-wave" signals from superdense objects are periodic variations in the collective gravitational shadow, created by dynamical systems of superdense objects.

\paragraph{Predictions.}
\begin{enumerate}
    \item \textbf{Dependence of Registered Signal Intensity on Viewing Angle.} In Pastalogy, these observed signals arise from changes in the mutual shielding of superdense objects in binary or multiple systems as they orbit. Consequently, the intensity of the registered effect should strongly depend on the viewing angle. A maximum signal would be recorded when observing superdense binary systems "edge-on" (i.e., from within their orbital plane), where the effect of mutual occultation and its variations are most pronounced. When observed "pole-on" (perpendicular to the orbital plane) or at other angles, the signal intensity would be significantly lower, potentially to the point of complete absence or extremely weak manifestation, as the occultation effect would be minimal or null. This differs from GR, where gravitational waves propagate in all directions and their intensity has a specific radiation pattern \cite{maggiore2007-gw}, but is not caused by direct occultation.
    \item \textbf{"Merger Precursors" and Precise Forecasting.} Precursors are possible for systems observed edge-on or close to such an orientation. It is essential that the superdense bodies pass in front of each other. In the late stages of the inspiral of two bodies in a binary system, when a significant geometric overlap of their gravitational shadows begins, Pastalogy predicts the possibility of registering specific "merger precursors." Instead of a continuously increasing quasi-periodic signal, Pastalogy suggests the appearance of "singular" small oscillations or "jolts" in the gravitational pressure. These oscillations would be registered with significant but progressively shortening time intervals between them, as the bodies of the binary system draw ever closer and their relative velocity increases. The jolts should become shorter and more powerful over time. Existing event registration records should be thoroughly examined for the presence of such single jolts. They may be extremely rare and faint, while having a long duration and slow rise and fall, which would explain why they have not been detected yet, but their existence is a key confirmation of Pastalogy. If jolts from different systems can be detected, they might at first appear to be randomly distributed in time; however, a detailed analysis would allow them to be separated by object and enable highly accurate predictions of the next jolts from each specific system. Such a pattern would allow for:
    \begin{itemize}
        \item \textbf{Precisely predicting merger times:} by analyzing the dynamics of the shortening intervals between "jolts," it would be possible to calculate the remaining time until the event's culmination with high accuracy.
        \item \textbf{Forecasting the strength of future bursts:} the characteristics of these precursors could carry information about the masses and velocities of the inspiraling bodies, allowing for predictions of the main gravitational pressure burst's power.
        \item \textbf{Targeted observation:} these precursors would allow astronomers to pre-emptively aim telescopes and detectors to observe merger events in a targeted manner, rather than discovering them retrospectively from a signal that has already occurred, as is often the case with current merger registrations.
    \end{itemize}
\end{enumerate}

These predictions from Pastalogy open new avenues for astrophysical observations and could potentially be tested with highly sensitive detectors designed to register such gravitational perturbations.

\subsection{Predictions for Modified Slit Experiments}\label{subsec:slit-experiments}

The explanation of wave-particle duality in Pastalogy (see Section~\ref{subsec:wave-particle-duality}) as an interaction between a discrete cluster of pastons (the particle) and a real, physical wave front in the medium of background pastons leads to a specific, testable prediction for a modified double-slit experiment, which differs significantly from the interpretations of standard quantum mechanics.

\paragraph{Key Postulates of Pastalogy.}
\begin{itemize}
    \item A particle is always a discrete, localized cluster of pastons.
    \item The "wave" is a real perturbation in the ubiquitous background of pastons, generated by the particle's motion. This wave front passes through all available slits and interferes, yet it is not itself registered by detectors and it interferes with other fronts without destroying them.
    \item A particle passing through one of the slits is "guided" toward the maxima of the interference pattern formed by the background wave front.
\end{itemize}

\paragraph{Prediction.} Let us consider a modified double-slit experiment in which the particle source (e.g., photons or electrons \cite{tonomura1989-electron}) is configured such that all particles are guaranteed to pass exclusively through only one of the two slits. This can be achieved by alternately opening one slit at a time and recording particles passing only through the active slit. Thus, each particle passes through only one slit and cannot "interfere with itself" or with other particles passing through the second slit, as is assumed in standard quantum mechanics. In Pastalogy, however, the background wave front generated by the particle's motion passes unimpeded through both slits and interferes, creating a characteristic interference pattern behind them. Since the particle is "guided" by this wave front, Pastalogy predicts that even with all particles guaranteed to pass through only one slit, a full two-slit interference pattern, centered opposite the active slit, will still be observed on the detector.

\paragraph{Prediction of Standard Quantum Mechanics.} In standard quantum mechanics, the state of a particle in a double-slit experiment is described as a superposition of paths through both slits. The interference pattern arises from the interaction of these paths. However, if the particle's path is unambiguously determined (e.g., the particle passes through only one slit, and the second slit makes no contribution to the wave function), the interference effects vanish. As a result, a single-slit diffraction pattern is observed on the detector---a single bright band corresponding to the projection of the open slit, without the characteristic interference maxima and minima.

Formally, this is expressed as follows. In standard quantum mechanics, the amplitude of the state in the screen region is given by the sum of two components:
\begin{equation}
    \Psi(x) = \Psi_1(x) + \Psi_2(x).
\end{equation}
The probability of registering a particle is then
\begin{equation}
    P(x) = |\Psi(x)|^2 = |\Psi_1(x)|^2 + |\Psi_2(x)|^2 + \Psi_1^*(x)\Psi_2(x) + \Psi_2^*(x)\Psi_1(x),
\end{equation}
where the last two terms describe the interference terms.

In the modified experiment, with particles guaranteed to pass through only one slit, we have
\begin{equation}
    \Psi_2(x) = 0 \Rightarrow \Psi(x) = \Psi_1(x), \quad P(x) = |\Psi_1(x)|^2.
\end{equation}

Thus, standard quantum mechanics predicts a \emph{single-slit diffraction pattern} without interference.

\paragraph{Comparison and Falsifiability.} Thus, the predictions of Pastalogy and standard quantum mechanics diverge fundamentally:
\begin{itemize}
    \item Standard quantum mechanics: when particles pass through only one slit, a single-slit diffraction pattern is observed.
    \item Pastalogy: when particles pass through only one slit, a two-slit interference pattern is observed.
\end{itemize}
This modified double-slit experiment constitutes a falsifying test that allows for a clear distinction between the predictions of Pastalogy and standard quantum mechanics. The observation of an interference pattern under conditions of guaranteed single-slit passage would provide strong support for Pastalogy, whereas a diffraction pattern would confirm standard quantum mechanics.

\subsection{Explanation of Existing Astronomical Anomalies}\label{subsec:astronomical-anomalies}

Pastalogy offers testable predictions for astronomical observations that are perceived as anomalies in standard cosmology, requiring additional hypotheses. Such observations include the existence of black holes with masses exceeding expectations, objects with extremely high redshifts, long-duration flares like "Scary Barbie," and objects with anomalously low redshifts. These phenomena are naturally explained within Pastalogy as consequences of local Big Bangs and are amenable to observational verification.

\paragraph{Predictions.}
\begin{enumerate}
    \item \textbf{Black holes with anomalously large masses in the observable Universe.} In standard cosmology, the formation of supermassive black holes at high redshifts (\( z > 6 \)) is problematic \cite{larson2023-jwst-bh}, a tension exacerbated by recent discoveries of early black hole populations by JWST \cite{maiolino2024-jades-bh}. Pastalogy predicts that such black holes are direct fragments of a local Big Bang (see Section~\ref{subsubsec:big-bang}), the collision of two giant teftels (matured black holes) that initiated the observable part of the Universe, when the FSM "smashed one meatball against another." Pastalogy predicts that such black holes will be accompanied by other dense structures (e.g., compact star clusters) with an anomalously high density in their vicinity.    
    \item \textbf{Mature galaxies at critically high redshifts.} The observation of mature galaxies at \( z > 10 \) \cite{labbe2023-jwst-galaxies} and even \( z \approx 14 \) \cite{carniani2024-z14} contradicts standard \(\Lambda\)CDM cosmology, which requires significant adjustments to explain their abundance and mass \cite{lu2024-jwst-lcdm-tension}. Pastalogy predicts that most such galaxies exhibit high metallicity and complex morphology, resulting from their formation from the material of a local Big Bang. Redshift is interpreted as a combination of the Doppler effect and photon aging (see Section~\ref{subsec:light-aging}), and an infinite and eternal universe eliminates the time constraints on structure formation.
    \item \textbf{Long-duration, non-fading flares from "Minor Bangs."} Pastalogy predicts that head-on collisions of ordinary black holes in large voids, where the low density of objects allows for a direct approach without an orbital dance, generate "Minor Bangs." These events manifest as long-duration (several years), non-fading flares, gradually expanding in space, without a detectable host galaxy. A potential example is "Scary Barbie" (AT 2021lwx) \cite{subrayan2023-barbie}, with an energy of \( 1.5 \times 10^{53} \) erg and a duration of \( >3 \) years, which could be the result of such a collision rather than a standard tidal disruption event.
    \item \textbf{Objects with anomalously low redshift.} Pastalogy predicts that in some directions, rare cosmic bodies may be found with a redshift that is unnaturally low for their distance (estimated, for example, by luminosity or angular size) \cite{arp1987-quasars}, as a consequence of a different local Big Bang in an infinite universe (see Section~\ref{subsubsec:big-bang}). This is due to the partial compensation of the redshift from photon aging (see Section~\ref{subsec:light-aging}) by a blue Doppler shift from motion toward the Earth.
\end{enumerate}
 
\paragraph{Methods of Verification and Falsification.} The predictions of Pastalogy can be tested using telescopes such as JWST or ZTF. For black holes (point 1), observations should reveal a correlation of their location with dense structures in their vicinity, indicating an origin from a local Big Bang. If black holes are distributed uniformly without such correlations, this would weaken Pastalogy. For galaxies at \( z > 10 \) (point 2), spectroscopy should show that most of them have high metallicity and a complex structure. If most galaxies turn out to be simple and have low metallicity, this would weaken Pastalogy. For "Minor Bangs" (point 3), observations in voids should reveal long-duration flares without a host galaxy, such as "Scary Barbie." If such flares fade like typical tidal disruption events or gamma-ray bursts, this would weaken Pastalogy. For objects with anomalously low redshift (point 4), spectroscopy should reveal rare objects with a redshift smaller than expected for their distance. If such objects are not detected over a long period, this would weaken Pastalogy, considering their rarity. The detection of the predicted characteristics would serve as strong confirmation of Pastalogy.

\section{Discussion and Conclusion}\label{sec:discussion-conclusion}

\subsection{Strengths of Pastalogy}\label{subsec:strengths}

As a new axiomatic theory of the universe, Pastalogy possesses a number of fundamental advantages that favorably distinguish it from existing scientific paradigms and lend it special value in the search for a complete and logically grounded description of the universe:

\begin{itemize}
    \item \textbf{Logical Parsimony:} One of the most outstanding merits of Pastalogy is its ability to construct a complete and comprehensive picture of the world starting from a single fundamental axiomatic assumption: that matter is divisible, but not infinitely so. This ultimate simplicity of its foundational principles makes Pastalogy a model of logical economy in scientific construction.
    
    \item \textbf{Internal Consistency and Coherence:} All propositions of Pastalogy, from the nature of fundamental particles (pastons) to the explanation of cosmological phenomena, are deductively derived from this single axiomatic principle. This ensures the theory's exceptional internal consistency and conceptual coherence, where every part is logically connected to the others.
    
    \item \textbf{Elegance of Explanations:} Pastalogy offers elegant and intuitively understandable explanations for complex physical phenomena, reducing them to simple mechanical principles of paston interactions. Gravity as pressure, light as clusters of pastons, and wave-particle duality as the interaction of a particle with a physical wave front in a medium---all these concepts are simple, clear, and do not require the introduction of abstract mathematical constructs without a physical analog.
    
    \item \textbf{Elimination of "Supernatural" or Unfounded Entities:} One of the key advantages of Pastalogy is its ability to eliminate from the fundamental description of the universe a number of entities that in standard physics often appear "supernatural" or are introduced \textit{ad hoc} to explain observations. Pastalogy dispenses with fundamental fields (replacing them with the mechanical interaction of pastons), "dark energy" (as it does not postulate an accelerated expansion of space), and the non-intuitive "collapse of the wave function" (explaining it as the disruption of a real wave front).
    
    \item \textbf{High Explanatory and Predictive Power:} Despite its axiomatic simplicity, Pastalogy demonstrates high explanatory power for a wide range of observed macroscopic and microscopic phenomena, including gravity, light, wave-particle duality, and astronomical anomalies. Moreover, it offers a series of specific, testable predictions that differ from those of standard physics (e.g., regarding the non-identity of gravitational and inertial mass for superdense objects, the specifics of gravitational perturbations, and the behavior of particles in modified slit experiments), which opens a path for its empirical verification.
\end{itemize}

\subsection{The Place of Pastalogy in Science}\label{subsec:place-in-science}

Pastalogy occupies a unique position in the landscape of modern scientific knowledge, as it differs substantially from the traditional empirical approach characteristic of physics. In essence, Pastalogy is not physics in its empirical sense, which builds its theories starting from observations and experiments. Rather, Pastalogy is a fundamental axiomatic science whose goal is the deductive construction of the universe from foundational principles. It begins with a single, maximally simple axiomatic assumption and, through rigorous deduction, aims to derive all known physical phenomena and cosmological structures.

Its task is not so much to describe what is already observed (although it does so successfully) as to provide a logically consistent and self-contained picture of reality that flows from a minimal set of postulates. This approach aligns Pastalogy more with mathematical systems or the philosophical foundations of science, offering an alternative path to understanding the fundamental nature of the universe.

Nevertheless, Pastalogy does not seek to isolate itself from empirical science. On the contrary, its strength lies in its potential "convergence" with physics. This "convergence" will occur when the empirical data obtained by physics and the logical conclusions drawn by Pastalogy align and fully coincide. If the predictions of Pastalogy are experimentally confirmed, it will mark the moment when axiomatic construction and empirical observation merge, forming a single, complete, and deeply grounded picture of the universe. In such a case, Pastalogy could provide physics with the fundamental level of understanding that has hitherto remained beyond its reach, offering not just a description but an explanation for the causes of all observed phenomena.

\subsection{Future Research Directions}\label{subsec:future-directions}

Although Pastalogy already offers a coherent and internally consistent picture of the universe and explains a number of key phenomena and anomalies, its full development requires significant efforts in several key areas. These directions represent a roadmap for future theoretical and, potentially, experimental research:

\begin{itemize}
    \item \textbf{Development of a Formal Mathematical Apparatus:} At this stage, Pastalogy is presented primarily at a conceptual and qualitative level, demonstrating the logical coherence of its principles. However, to transition to quantitative predictions and rigorous scientific analysis, the development of a complete formal mathematical apparatus is critically important. This includes creating mathematical models that describe the dynamics of individual pastons, their collisions, the formation of stable aggregates, and the propagation of perturbations in the medium of background pastons. Such an apparatus would make it possible to calculate precise values for observable quantities and to conduct simulations of complex phenomena.
    
    \item \textbf{Detailed Theoretical Modeling of the Formation of Stable Complex Particles and the Derivation of Their Properties:} The existing tenets of Pastalogy explain the nature of "particles" as clusters of pastons. The next stage is the detailed theoretical modeling and derivation of the properties of specific stable complex particles, such as electrons, quarks, and neutrinos, from the behavior and configurations of pastons. This includes not only their mass but also such fundamental characteristics as electric charge (as a specific asymmetry in the motion of pastons), spin (as the internal rotation of a paston aggregate), and "color" (as a special configuration that determines the strong interactions). The successful derivation of these properties from the first principles of Pastalogy would be a colossal breakthrough.
    
    \item \textbf{Proposal of Specific Experimental Setups to Test Key Predictions:} Although this paper already contains a number of predictions that differ from standard physics (e.g., concerning the gravitational influence of superdense objects and the behavior of particles in modified slit experiments), future research should focus on developing specific, detailed designs for experimental setups capable of testing these predictions. This will require a deep analysis of technical feasibility, precision measurement requirements, and the development of innovative methodologies that can detect the subtle effects predicted by Pastalogy but ignored or explained differently within the standard model. Such experiments, even if they seem futuristic, are the ultimate goal and the criterion for the theory's verifiability.
\end{itemize}

\subsection{Concluding Remarks}\label{subsec:conclusion}

This paper has presented the axiomatic theory of Pastalogy---an attempt to rethink the fundamental principles of physics and cosmology using a minimal number of assumptions. Starting from a single axiomatic principle---the divisibility of matter down to the smallest, indivisible pastons---Pastalogy offers a complete and internally consistent picture of the universe. Within its framework, phenomena such as gravity, light, wave-particle duality, and a range of astronomical anomalies are explained through the simple mechanical interactions of these basic elements.

Pastalogy challenges existing scientific paradigms by proposing, in place of abstract fields and probabilistic waves, a concrete physical reality consisting of discrete particles and their interactions. It calls for a re-examination of established notions of space, time, mass, and energy, suggesting they be replaced with concepts derivable from the behavior of pastons.

Ultimately, Pastalogy presents itself not merely as another hypothesis, but as a new paradigm for understanding the universe. It opens a path toward the creation of a unified theory of everything, based on logic and deduction, and offers specific, testable predictions that can become the decisive criterion of its truth. If Pastalogy withstands the scrutiny of future experiments and observations, it has the potential to forever change our view of the universe, restoring to physics an intuitive clarity and depth of understanding rooted in the most elementary principles of reality.

\backmatter

\section*{Acknowledgments}\label{sec:acknowledgements}

During the preparation of this paper, the author utilized a number of artificial intelligence tools. To improve the style, assist with LaTeX typesetting, and translate the manuscript into English, large language models (LLMs) were employed, including several publicly available systems as well as the author's own tool, AIS Agora. The author has thoroughly reviewed, edited, and assumes full responsibility for all assertions and the final text of the manuscript.

\begin{thebibliography}{99}

\bibitem{einstein1916}
Einstein, A. (1916). Die Grundlage der allgemeinen Relativitätstheorie. \textit{Annalen der Physik}, 354(7), 769-822.

\bibitem{bohr1928}
Bohr, N. (1928). The Quantum Postulate and the Recent Development of Atomic Theory. \textit{Nature}, 121(3050), 580-590.

\bibitem{rubin1980}
Rubin, V. C., Ford, W. K., Jr., \& Thonnard, N. (1980). Rotational properties of 21 Sc galaxies with a large range of luminosities and radii, from NGC 4605 (R = 4kpc) to UGC 2885 (R = 122kpc). \textit{The Astrophysical Journal}, 238, 471-487.

\bibitem{riess1998}
Riess, A. G., et al. (1998). Observational evidence from supernovae for an accelerating universe and a cosmological constant. \textit{The Astronomical Journal}, 116(3), 1009-1038.

\bibitem{euclid-elements-heath}
Heath, T. L. (Trans. \& ed.). (1956). \textit{The Thirteen Books of Euclid's Elements} (2nd ed., 3 Vols.). New York: Dover Publications.

\bibitem{wiener1948}
Wiener, N. (1948). \textit{Cybernetics: Or Control and Communication in the Animal and the Machine}. New York: John Wiley \& Sons.

\bibitem{popper1959}
Popper, K. (1959). \textit{The Logic of Scientific Discovery}. London: Hutchinson.

\bibitem{kirk1983-democritus}
Kirk, G. S., Raven, J. E., \& Schofield, M. (1983). \textit{The Presocratic Philosophers: A Critical History with a Selection of Texts} (2nd ed.). Cambridge: Cambridge University Press.

\bibitem{leibniz1989-monadology}
Leibniz, G. W. (1989). The Monadology. In R. Ariew \& D. Garber (Eds. \& Trans.), \textit{Philosophical Essays} (pp. 213-225). Indianapolis: Hackett Publishing Company.

\bibitem{henderson2006}
Henderson, B. (2006). \textit{The Gospel of the Flying Spaghetti Monster}. Villard Books.

\bibitem{occam-razor-sep}
Baker, A. (2022). Simplicity. In E. N. Zalta (Ed.), \textit{The Stanford Encyclopedia of Philosophy} (Winter 2022 ed.). Metaphysics Research Lab, Stanford University.

\bibitem{pdg2024}
Navas, S., et al. (Particle Data Group). (2024). Review of Particle Physics. \textit{Physical Review D}, 110(3), 030001.

\bibitem{feynman-lectures-v1}
Feynman, R. P., Leighton, R. B., \& Sands, M. (2011). \textit{The Feynman Lectures on Physics, Vol. 1: The New Millennium Edition}. New York: Basic Books.

\bibitem{mtw1973}
Misner, C. W., Thorne, K. S., \& Wheeler, J. A. (1973). \textit{Gravitation}. San Francisco: W. H. Freeman and Company.

\bibitem{lesage-edwards2014}
Edwards, M. R. (2014). Pushing Gravity: A Review of Le Sage-type Theories. \textit{Studies in History and Philosophy of Science Part B}, 47, 1-15.

\bibitem{einstein1920-relativity}
Einstein, A. (2015). \textit{Relativity: The Special and the General Theory}. Princeton, NJ: Princeton University Press.

\bibitem{shapiro-teukolsky1983}
Shapiro, S. L., \& Teukolsky, S. A. (1983). \textit{Black Holes, White Dwarfs, and Neutron Stars: The Physics of Compact Objects}. New York: Wiley-Interscience.

\bibitem{newton1687-principia}
Newton, I. (1999). \textit{The Principia: Mathematical Principles of Natural Philosophy} (I. B. Cohen \& A. Whitman, Trans.). Berkeley, CA: University of California Press. (Original work published 1687).

\bibitem{halliday2013-fundamentals}
Halliday, D., Resnick, R., \& Walker, J. (2013). \textit{Fundamentals of Physics} (10th ed.). Hoboken, NJ: John Wiley \& Sons.

\bibitem{newman2013-computational}
Newman, M. (2013). \textit{Computational Physics}. CreateSpace Independent Publishing Platform.

\bibitem{ghez2008-sgrA}
Ghez, A. M., et al. (2008). Measuring Distance and Properties of the Milky Way's Central Supermassive Black Hole with Stellar Orbits. \textit{The Astrophysical Journal}, 689(2), 1044-1062.

\bibitem{schneider2006-lensing}
Schneider, P., Kochanek, C. S., \& Wambsganss, J. (2006). \textit{Gravitational Lensing: Strong, Weak and Micro}. Springer-Verlag Berlin Heidelberg.

\bibitem{lisa-consortium2017}
Amaro-Seoane, P., et al. (LISA Consortium). (2017). Laser Interferometer Space Antenna. \textit{arXiv preprint arXiv:1702.00786}.

\bibitem{ligo2016-detection}
Abbott, B. P., et al. (LIGO Scientific Collaboration and Virgo Collaboration). (2016). Observation of Gravitational Waves from a Binary Black Hole Merger. \textit{Physical Review Letters}, 116(6), 061102.

\bibitem{maxwell1873-treatise}
Maxwell, J. C. (1873). \textit{A Treatise on Electricity and Magnetism}. Oxford: Clarendon Press.

\bibitem{debroglie1924-thesis}
de Broglie, L. (1924). Recherches sur la théorie des quanta (Researches on the quantum theory). \textit{Annales de Physique}, 10(2), 22-128.

\bibitem{hubble1929}
Hubble, E. (1929). A relation between distance and radial velocity among extra-galactic nebulae. \textit{Proceedings of the National Academy of Sciences}, 15(3), 168-173.

\bibitem{goldhaber2001-dilation}
Goldhaber, G., et al. (The Supernova Cosmology Project). (2001). Timescale Stretch Parameterization of Type Ia Supernova B-band Light Curves. \textit{The Astrophysical Journal}, 558(1), 359-368.

\bibitem{penzias1965-cmb}
Penzias, A. A., \& Wilson, R. W. (1965). A Measurement of Excess Antenna Temperature at 4080 Mc/s. \textit{The Astrophysical Journal}, 142, 419-421.

\bibitem{alpher1948-bigbang}
Alpher, R. A., Bethe, H., \& Gamow, G. (1948). The Origin of Chemical Elements. \textit{Physical Review}, 73(7), 803-804.

\bibitem{steinhardt2007-cyclic}
Steinhardt, P. J., \& Turok, N. (2007). \textit{Endless Universe: Beyond the Big Bang}. New York: Doubleday.

\bibitem{milgrom1983-mond}
Milgrom, M. (1983). A modification of the Newtonian dynamics as a possible alternative to the hidden mass hypothesis. \textit{The Astrophysical Journal}, 270, 365-370.

\bibitem{bertone2005-review}
Bertone, G., Hooper, D., \& Silk, J. (2005). Particle dark matter: Evidence, candidates and constraints. \textit{Physics Reports}, 405(5-6), 279-390.

\bibitem{bohm1952-interpretation}
Bohm, D. (1952). A Suggested Interpretation of the Quantum Theory in Terms of "Hidden Variables". I. \textit{Physical Review}, 85(2), 166-179.

\bibitem{maggiore2007-gw}
Maggiore, M. (2007). \textit{Gravitational Waves: Volume 1: Theory and Experiments}. Oxford: Oxford University Press.

\bibitem{tonomura1989-electron}
Tonomura, A., Endo, J., Matsuda, T., Kawasaki, T., \& Ezawa, H. (1989). Demonstration of single-electron buildup of an interference pattern. \textit{American Journal of Physics}, 57(2), 117-120.

\bibitem{larson2023-jwst-bh}
Larson, R. L., et al. (2023). A CEERS Discovery of an Overmassive AGN in a Low-mass Galaxy at z = 5.5. \textit{The Astrophysical Journal Letters}, 953(2), L29.

\bibitem{labbe2023-jwst-galaxies}
Labbé, I., et al. (2023). A population of red candidate massive galaxies ~600 Myr after the Big Bang. \textit{Nature}, 616(7956), 266-269.

\bibitem{subrayan2023-barbie}
Subrayan, B., et al. (2023). AT 2021lwx: the most luminous transient ever observed?. \textit{Monthly Notices of the Royal Astronomical Society}, 526(2), 1623-1634.

\bibitem{maiolino2024-jades-bh}
Maiolino, R., Scholtz, J., et al. (2024). A small and vigorous black hole in the early Universe. \textit{Nature}, 625, 477-480.

\bibitem{carniani2024-z14}
Carniani, S., Hainline, K., et al. (2024). Spectroscopic confirmation of two luminous galaxies at a redshift of ~14. \textit{Nature Astronomy}, 8, 1063-1070.

\bibitem{lu2024-jwst-lcdm-tension}
Lu, S., Frenk, C. S., et al. (2024). A comparison of pre-existing LambdaCDM predictions with the abundance of JWST galaxies at high redshift. \textit{arXiv preprint arXiv:2406.02672}.

\bibitem{arp1987-quasars}
Arp, H. (1987). \textit{Quasars, Redshifts and Controversies}. Berkeley: Interstellar Media.

\end{thebibliography}

\end{document}